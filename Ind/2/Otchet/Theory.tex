\section{Теория}
\par
Прежде чем приступать к реализации разбора выражения, будет хорошей идеей
ознакомиться с имеющейся теорией по данному вопросу.\\
{\bf Дерево разбора выражения} — в информатике это конечное,
помеченное, ориентированное дерево, в котором внутренние вершины сопоставлены
с (помечены) операторами языка, а листья — с соответствующими
операндами. Таким образом листья являются пустыми операторами и представляют
только переменные и константы. Деревья разбора используются в парсерах,
например, компиляторах языков программирования,
для промежуточного представления программы, которое затем используется в
качестве внутреннего представления компилятора или интерпретатора компьютерной
программы для оптимизации и генерации кода.\\
Также для разбора алгебраического выражения будет удобно использовать {\it
алгоритм сортировочной станции}.\\
{\bf Алгоритм сортировочной станции} - способ разбора математических выражений,
представленных в инфиксной нотации. Может быть использован для получения
вывода в виде обратной польской нотации или в виде абстрактного
синтаксического дерева. Алгоритм изобретен Эдсгером Дейкстрой и назван им
``алгоритм сортировочной станции'', поскольку напоминает действие
железнодорожной сортировочной станции.\\
Так же, как и вычисление значений выражений в обратной польской записи,
алгоритм работает при помощи стека. Инфиксная запись математических выражений
чаще всего используется людьми, ее примеры: $2+4$ и $3+6*(3-2)$. Для
преобразования в обратную польскую нотацию используется 2 очереди: входная и
выходная, и стек для хранения операторов, еще не добавленных в выходную
очередь. При преобразовании алгоритм считывает 1 символ и производит действия,
зависящие от данного символа.
\pagebreak
