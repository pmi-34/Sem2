% Это комментарий
% Комментарии начинаются с символа %
% Формат А4, стиль - отчет
\documentclass[a4paper,12pt]{report}
% Поля - 2 сантиметра с каждой стороны, без места на подшивку
\usepackage[left=2cm,right=2cm,
    top=2cm,bottom=2cm,bindingoffset=0cm]{geometry}
% Входная кодировка - UTF-8
\usepackage[utf8]{inputenc}
% Язык текста - русский и английский. Русский - основной
\usepackage[english,russian]{babel}
% Для команды \Box и других математических
\usepackage{amsfonts}
% Для вставки изображений
\usepackage{graphicx}
% Для вставки исходного кода
% \usepackage{listings}
% Настройки листингов
% \lstloadlanguages{[x86masm]Assembler}
% \lstset{
% language=[x86masm]Assembler,
% extendedchars=\true,
% inputencoding=utf8,
% commentstyle=\itshape,
% stringstyle=\bf,
% belowcaptionskip=5pt
% }
% Нумерованные списки в виде 1.1
\renewcommand{\labelenumii}{\arabic{enumi}.\arabic{enumii}.}
\renewcommand{\labelenumiii}{\arabic{enumi}.\arabic{enumii}.\arabic{enumiii}.}
\renewcommand{\labelenumiv}{\arabic{enumi}.\arabic{enumii}.\arabic{enumiii}.\arabic{enumiv}.}
% Первый абзац должен начинаться с красной строки
\usepackage{indentfirst}
\usepackage{float}
\restylefloat{figure}
% Мы не используем главы, поэтому подправим номера секций
\renewcommand{\thesection}{\arabic{section}.}
% Исправим досадную ошибку с неразрывным пробелом в UTF-8
\DeclareUnicodeCharacter{00A0}{~}
\begin{document}
% Титульная страница
\begin{titlepage}
\begin{center}

% Верх страницы

\textsc{\large \bf Федеральное агентство по образованию\\
ГОУ ВПО «Пермский государственный национально-исследовательский университет»\\
Кафедра прикладной математики и информатики}\\[1.5cm]

\vfill

% Заголовок
{ \Large \bfseries Индивидуальная работа \No4\\
``Графы и среда Delphi''\\
\em по дисциплине ``Основы программирования'' \\[0.4cm] }

% Автор и рецензент
\begin{flushright}
\begin{minipage}{0.5 \textwidth}
\begin{flushleft} \large
\emph{\bf Выполнил:} \\
Студент 1-го курса \\
механико-математического~ф-та \\
\textsc{Протопопов У.Г.},\\
группа ПМИ-0.\\
\emph{\bf Проверил:} \\
Доцент кафедры ПМИ ПГУ,\\
к.т.н {\underline{\hspace{50 pt}}} \textsc{Перескокова О.И}\\
\today
\end{flushleft}
\end{minipage}
\end{flushright}

\vfill

% Низ страницы
Пермь \the\year

\end{center}
\end{titlepage}


\setcounter{page}{2}
\tableofcontents
\pagebreak
\section{Постановка задачи}
\par
Написать программу, которая по заданной формуле строит дерево и производит
вычисления с его помощью. Формула задана в традиционной инфиксной записи, в
ней могут встречаться скобки, максимальная степень вложенности которых
ограничивается числом 10. Аргументами могут быть целые числа и переменные,
задаваемые однобуквенными именами. Допустимы операции {\it +, -, *, /}.
Унарный минус допустим. С помощью построенного дерева формулы упростить
формулу, заменив в ней все поддеревья, соответствующие формулам $ (f_1 \pm f_2
)*f_3$ и $f_1*(f_2 \pm f_3)$ на поддеревья, соответствующие формулам $f_1f_3
\pm f_2f_3$ и $f_1f_2 \pm f_1f_3$.
\pagebreak


\section{Теория}
\par
Прежде чем приступать к реализации разбора выражения, будет хорошей идеей
ознакомиться с имеющейся теорией по данному вопросу.\\
{\bf Дерево разбора выражения} — в информатике это конечное,
помеченное, ориентированное дерево, в котором внутренние вершины сопоставлены
с (помечены) операторами языка, а листья — с соответствующими
операндами. Таким образом листья являются пустыми операторами и представляют
только переменные и константы. Деревья разбора используются в парсерах,
например, компиляторах языков программирования,
для промежуточного представления программы, которое затем используется в
качестве внутреннего представления компилятора или интерпретатора компьютерной
программы для оптимизации и генерации кода.\\
Также для разбора алгебраического выражения будет удобно использовать {\it
алгоритм сортировочной станции}.\\
{\bf Алгоритм сортировочной станции} - способ разбора математических выражений,
представленных в инфиксной нотации. Может быть использован для получения
вывода в виде обратной польской нотации или в виде абстрактного
синтаксического дерева. Алгоритм изобретен Эдсгером Дейкстрой и назван им
``алгоритм сортировочной станции'', поскольку напоминает действие
железнодорожной сортировочной станции.\\
Так же, как и вычисление значений выражений в обратной польской записи,
алгоритм работает при помощи стека. Инфиксная запись математических выражений
чаще всего используется людьми, ее примеры: $2+4$ и $3+6*(3-2)$. Для
преобразования в обратную польскую нотацию используется 2 очереди: входная и
выходная, и стек для хранения операторов, еще не добавленных в выходную
очередь. При преобразовании алгоритм считывает 1 символ и производит действия,
зависящие от данного символа.
\pagebreak

\section{Определение идеи алгоритма, выбор методов решения и структур данных}
\par
Перед программой стоит задача - считать из файла выражение, преобразовать его
в виде дерева, упростить его и вывести в терминал.
\par
Для начала определимся с типами данных. По условию задачи необходимо
использовать в качестве внутреннего представления дерево. Определим тип запись
с полями, равными типу элемента (константа, переменная, оператор), значению
элемента (код символа - для операции и переменной и просто число - для целого
числа). Также структура содержит три поля ссылочного типа - два для
формирования дерева и одно - для формирования обратной польской записи.
\par
Ввод данных удобно организовать через файл; также стоит предусмотреть
возможность ввода строки с клавиатуры для удобства вычисления одиночного
выражения.
\par
Программа будет состоять из трех модулей - модуль ввода данных, модуль их
преобразования и модуль вывода. В модуле ввода центральное место займет
анализатор выражений, построенный по принципу сортировочной станции; в
модуле вывода - симметричный ему алгоритм
конвертации внутреннего представления в строку. В модуле преобразования
будет рекурсивная процедура, выполняющая указанные в задаче упрощения.
\par
В качестве отдельного Unit-файла я выделил операции работы со стеком и
подпрограмму обработки ошибок, что позволило удобно использовать эти процедуры
в основной программе.
\par
Пару слов насчет интерфейса. Поскольку основной ввод/вывод организуется через
файл, у пользователя необходимо запросить только лишь его имя (и проверить на
существование). Но в случае ошибки было бы неплохо указать, где именно
споткнулась программа (а такое место может быть только одно - анализатор
выражений), и в любом случае - вывести какое-нибудь сообщение по окончании
работы программы (``Все хорошо, хозяин'' или ``Явка провалена из-за ошибки
резидента'').
\pagebreak


\section{Детальное описание алгоритма}
\DeclareRobustCommand{\l}[1]{
\item {[}{\it #1}{]} 
}
Алгоритм состоит из следующих основных процедур:
\begin{itemize}
\item TMainForm.FormCreate - задает начальные параметры программы, считывая
их, если это возможно, из входного файла
\item TMainForm.FormPaint - прорисовывает в окне программы все основные
элементы, как-то поле, коня и траву
\item TMainForm.FieldPaintBoxMouseDown - обработка события нажатия кнопки мыши
на поле
\item BuildHorseMatrix2 - построение матрицы графа, инициализация + прямой ход
алгоритма
\item ReverseWave - обратный ход алгоритма 
\item TMainForm.FormClose - освобождение ресурсов и запись результата в
выходной файл
\end{itemize}

Детальное описание:
\begin{enumerate}
\item При запуске программы вызывается TMainForm.FormCreate. Она подгружает
графические ресурсы с диска и пытается прочитать начальные значения параметров
из входного файла. Если считывание не удалось, или во время него произошла
ошибка, параметрам присваиваются значения по умолчанию.
\item Основной цикл работы программы включает в себя два главных компонента:
отображение результата работы алгоритма и обработку действий пользователя
\begin{enumerate}
\item За отображение результата отвечает процедура TMainForm.FormPaint. Она
прорисовывает в окне прогаммы необходимые компоненты:
\begin{enumerate}
\item Сначала рисуется поле $NxN$ клеток-квадратов, клетки раскрашены в цвет
шахматной доски
\item Затем поверх поля отображается текущая позиция коня и травы, конь и
трава - спрайты, подгружаемые процедурой FormCreate
\item После этого рисуется текущий путь коня, начиная с его позиции и
заканчивая позицией, на которой размещена трава. Путь отображается
последовательностью символов ``галочка'', этот символ также является спрайтом
\item В самом конце производится проверка, нужно ли отображать матрицу
промежуточных результатов; если да - поверх полученного ранее изображения
рисуется матрица (для каждой клетки - расстояние от нее до клетки с конем)
\end{enumerate}
\item За взаимодействие с пользователем отвечает процедура
TMainForm.FieldPaintBoxMouseDown:
\begin{enumerate}
\item Если произошел щелчок левой кнопкой мыши, вычисляем новые координаты
коня, и, если они допустимы (не выходят за пределы поля), переносим его на
новую позицию
\item Если произошел щелчок правой кнопкой, то вычисляем новые координаты
травы и, соответственно, при их корректности переносим траву
\item По окончании переноса коня и травы заново запускаем алгоритм и форсируем
перерисовку
\end{enumerate}
\item Также на форме имеется элемент для ввода размерности поля. При его
изменении меняется размерность поля, а также корректируются координаты коня и
травы, если они выходят за границы поля новой размерности. По окончании
обработки события форсируется обновление данных и перерисовка поля.
\item По завершении работы программы результирующие данные (количество ходов
коня и координаты клеток, по которым пройдет его маршрут) записываются в
результирующий файл.
\end{enumerate}
\end{enumerate}

\pagebreak


\section{Тестирование программы}
\newcounter{testcnt}
\DeclareRobustCommand{\test}[3]{
  \addtocounter{testcnt}{1}
  \par
  Тест \arabic{testcnt}: {#1}\\
  {\it Входные данные:} {#2}\\
  {\it Выходные данные:} {#3}\\
  {\it Ожидаемый результат:} {#3}\\
  \begin {figure}[H] 
  \centering 
  \includegraphics[scale=0.5]{test\arabic{testcnt}.png} 
  \caption{Скриншот} 
  \label{fig:scr\arabic{testcnt}} 
  \end {figure}
}
\par
Тесты для программы были разбиты на несколько групп таким образом, чтобы
покрыть одновременно максимальное количество типов входных данных и вариантов
выполнения тела программы.\\
%\test{}{}{}{}
\test{Результат выполнения программы с нормальными входными данными}{Смотри
рисунок}{Корректные выходные данные}
\test{Выполнение программы с корректными входными данными}{Смотри предыдущий
рисунок}{Правильное отображение элементов}
\test{Обработка ситуации отсутствия входного файла}{Входной файл
отсутствует}{Всем параметрам присваивается значение по умолчанию}
\test{Некорректные входные данные}{Файл с некорректными данными}{Всем
параметрам присваиваются значения по умолчанию}
\test{Перемещение элементов}{Пользователь кликом мыши перемещает коня и
траву}{Конь и трава корректно перемещаются, путь пересчитывается}
\test{Динамическое изменение размера поля на больший}{Пользователь задает новый размер
поля}{Поле меняет размер, конь и трава не перемещаются; возможно,
пересчитывается путь}
\test{Динамическое изменение размера поля на меньший}{Пользователь задает
новый размер поля, конь и/или трава в него не помещаются}{Конь и/или трава
меняют свое положение на допустимые}
\pagebreak

\section{Заключение}
\par
В итоге данной индивидуальной работы нам удалось написать программу, которая
решает поставленную перед ней задачу. Кроме того, мы познакомились с
принципами разработки и тестирования ПО, научились применять их во благо.
\pagebreak

\section{Текст программы}
Ниже приведен полный исходный текст программы, созданной в текстовом 
редакторе Vim 7.4. Программа успешно собирается с использованием Lazarus IDE
1.2.2
для 64-разрядного Unix.\\
\fbox{Основная программа:}\\
% Generator: GNU source-highlight, by Lorenzo Bettini, http://www.gnu.org/software/src-highlite
\noindent
\mbox{}unit\ Unit1; \\
\mbox{} \\
\mbox{}\textit{\{\$mode\ objfpc\}\{\$H+\}} \\
\mbox{} \\
\mbox{}interface \\
\mbox{} \\
\mbox{}\textbf{uses} \\
\mbox{}\ \ Classes,\ SysUtils,\ FileUtil,\ Forms,\ Controls,\ Graphics,\ Dialogs,\ Spin, \\
\mbox{}\ \ StdCtrls,\ ExtCtrls,\ LCLIntf,\ Menus,\ GraphWork,\ unit2; \\
\mbox{} \\
\mbox{}\textbf{type} \\
\mbox{} \\
\mbox{}\ \ \textit{\{\ TMainForm\ \}} \\
\mbox{} \\
\mbox{}\ \ TMainForm\ =\ \textbf{class}(TForm) \\
\mbox{}\ \ \ \ ApplySizeButton:\ TButton; \\
\mbox{}\ \ \ \ GrassCoordLabel:\ TLabel; \\
\mbox{}\ \ \ \ Label2:\ TLabel; \\
\mbox{}\ \ \ \ HorseCoordLabel:\ TLabel; \\
\mbox{}\ \ \ \ MainMenu1:\ TMainMenu; \\
\mbox{}\ \ \ \ ExitItem:\ TMenuItem; \\
\mbox{}\ \ \ \ AboutItem:\ TMenuItem; \\
\mbox{}\ \ \ \ ParamItem:\ TMenuItem; \\
\mbox{}\ \ \ \ TimeLabel:\ TLabel; \\
\mbox{}\ \ \ \ MovesLabel:\ TLabel; \\
\mbox{}\ \ \ \ ShowDistCheckBox:\ TCheckBox; \\
\mbox{}\ \ \ \ Label1:\ TLabel; \\
\mbox{}\ \ \ \ FieldPaintBox:\ TPaintBox; \\
\mbox{}\ \ \ \ SizeSpinEdit:\ TSpinEdit; \\
\mbox{}\ \ \ \ \textbf{procedure}\ \textbf{AboutItemClick}(Sender:\ TObject); \\
\mbox{}\ \ \ \ \textbf{procedure}\ \textbf{ApplySizeButtonClick}(Sender:\ TObject); \\
\mbox{}\ \ \ \ \textbf{procedure}\ \textbf{ExitItemClick}(Sender:\ TObject); \\
\mbox{}\ \ \ \ \textbf{procedure}\ \textbf{FieldPaintBoxMouseDown}(Sender:\ TObject;\ Button:\ TMouseButton; \\
\mbox{}\ \ \ \ \ \ Shift:\ TShiftState;\ X,\ Y:\ Integer); \\
\mbox{}\ \ \ \ \textbf{procedure}\ \textbf{FormClose}(Sender:\ TObject;\ \textbf{var}\ CloseAction:\ TCloseAction); \\
\mbox{}\ \ \ \ \textbf{procedure}\ \textbf{FormCreate}(Sender:\ TObject); \\
\mbox{}\ \ \ \ \textbf{procedure}\ \textbf{FormPaint}(Sender:\ TObject); \\
\mbox{}\ \ \ \ \textbf{procedure}\ CalcPath; \\
\mbox{}\ \ \ \ \textbf{procedure}\ \textbf{ShowDistCheckBoxChange}(Sender:\ TObject); \\
\mbox{}\ \ private \\
\mbox{}\ \ \ \ \textit{\{\ private\ declarations\ \}} \\
\mbox{}\ \ \ \ \textit{//\ Размер\ поля\ в\ клетках} \\
\mbox{}\ \ \ \ ChessSize\ :\ integer; \\
\mbox{}\ \ \ \ Horse,\ Grass,\ Check\ :\ TPicture; \\
\mbox{}\ \ \ \ HX,\ HY\ :\ integer; \\
\mbox{}\ \ \ \ GX,\ GY\ :\ integer; \\
\mbox{}\ \ \ \ M\ :\ Graph; \\
\mbox{}\ \ \ \ Path\ :\ IArray; \\
\mbox{}\ \ \ \ AboutBox\ :\ TAboutBox; \\
\mbox{}\ \ public \\
\mbox{}\ \ \ \ \textit{\{\ public\ declarations\ \}} \\
\mbox{}\ \ \textbf{end}; \\
\mbox{} \\
\mbox{}\textit{//\ Размер\ клетки\ поля\ в\ пикселях} \\
\mbox{}\textbf{const} \\
\mbox{}\ \ CellSize\ =\ 40; \\
\mbox{}\ \ INFILE\ =\ \texttt{'input.txt'}; \\
\mbox{}\ \ OUTFILE\ =\ \texttt{'output.txt'}; \\
\mbox{} \\
\mbox{}\textbf{var} \\
\mbox{}\ \ MainForm:\ TMainForm; \\
\mbox{} \\
\mbox{}implementation \\
\mbox{} \\
\mbox{}\textit{\{\$R\ *.lfm\}} \\
\mbox{} \\
\mbox{}\textit{\{\ TMainForm\ \}} \\
\mbox{} \\
\mbox{}\textbf{procedure}\ TMainForm.\textbf{FormCreate}(Sender:\ TObject); \\
\mbox{}\textbf{var} \\
\mbox{}\ \ f\ :\ \textbf{text}; \\
\mbox{}\textbf{begin} \\
\mbox{}\ \ FieldPaintBox.Width\ :=\ SizeSpinEdit.MaxValue*CellSize; \\
\mbox{}\ \ FieldPaintBox.Height\ :=\ SizeSpinEdit.MaxValue*CellSize; \\
\mbox{} \\
\mbox{}\ \ Horse\ :=\ TPicture.Create; \\
\mbox{}\ \ Horse.\textbf{LoadFromFile}(\texttt{'kon2.png'}); \\
\mbox{} \\
\mbox{}\ \ Grass\ :=\ TPicture.Create; \\
\mbox{}\ \ Grass.\textbf{LoadFromFile}(\texttt{'gra2.png'}); \\
\mbox{} \\
\mbox{}\ \ Check\ :=\ TPicture.Create; \\
\mbox{}\ \ Check.\textbf{LoadFromFile}(\texttt{'che2.png'}); \\
\mbox{} \\
\mbox{}\ \ try \\
\mbox{}\ \ \ \ \textit{//\ Считываем\ данные\ из\ файла} \\
\mbox{}\ \ \ \ \textbf{AssignFile}(f,\ INFILE); \\
\mbox{}\ \ \ \ \textbf{Reset}(F); \\
\mbox{}\ \ \ \ \textbf{Readln}(F,\ ChessSize); \\
\mbox{}\ \ \ \ \textbf{if}\ (ChessSize\ \textless{}\ 5)\ \textbf{or}\ (ChessSize\ \textgreater{}\ 20)\ \textbf{then} \\
\mbox{}\ \ \ \ \ \ ChessSize\ :=\ 5; \\
\mbox{}\ \ \ \ SizeSpinEdit.Value\ :=\ ChessSize; \\
\mbox{}\ \ \ \ \textbf{Readln}(F,\ HX,\ HY); \\
\mbox{}\ \ \ \ \textbf{if}\ (HX\ \textless{}\ 1)\ \textbf{or}\ (HX\ \textgreater{}\ ChessSize)\ \textbf{then} \\
\mbox{}\ \ \ \ \ \ HX\ :=\ 0 \\
\mbox{}\ \ \ \ \textbf{else} \\
\mbox{}\ \ \ \ \ \ \textbf{dec}(HX); \\
\mbox{}\ \ \ \ \textbf{if}\ (HY\ \textless{}\ 1)\ \textbf{or}\ (HY\ \textgreater{}\ ChessSize)\ \textbf{then} \\
\mbox{}\ \ \ \ \ \ HY\ :=\ 0 \\
\mbox{}\ \ \ \ \textbf{else} \\
\mbox{}\ \ \ \ \ \ \textbf{dec}(HY); \\
\mbox{}\ \ \ \ \textbf{Readln}(F,\ GX,\ GY); \\
\mbox{}\ \ \ \ \textbf{if}\ (GX\ \textless{}\ 1)\ \textbf{or}\ (GX\ \textgreater{}\ ChessSize)\ \textbf{then} \\
\mbox{}\ \ \ \ \ \ GX\ :=\ 0 \\
\mbox{}\ \ \ \ \textbf{else} \\
\mbox{}\ \ \ \ \ \ \textbf{dec}(GX); \\
\mbox{}\ \ \ \ \textbf{if}\ (GY\ \textless{}\ 1)\ \textbf{or}\ (GY\ \textgreater{}\ ChessSize)\ \textbf{then} \\
\mbox{}\ \ \ \ \ \ GY\ :=\ 0 \\
\mbox{}\ \ \ \ \textbf{else} \\
\mbox{}\ \ \ \ \ \ \textbf{dec}(GY); \\
\mbox{}\ \ \ \ \textbf{CloseFile}(F); \\
\mbox{}\ \ except \\
\mbox{}\ \ \ \ ChessSize\ :=\ SizeSpinEdit.MinValue; \\
\mbox{}\ \ \ \ HX\ :=\ 0; \\
\mbox{}\ \ \ \ HY\ :=\ 0; \\
\mbox{} \\
\mbox{}\ \ \ \ GX\ :=\ ChessSize\ -\ 1; \\
\mbox{}\ \ \ \ GY\ :=\ ChessSize\ -\ 1; \\
\mbox{}\ \ \textbf{end}; \\
\mbox{} \\
\mbox{}\ \ CalcPath; \\
\mbox{}\ \ AboutBox\ :=\ TAboutBox.\textbf{Create}(self); \\
\mbox{}\textbf{end}; \\
\mbox{} \\
\mbox{}\textbf{procedure}\ TMainForm.\textbf{FormPaint}(Sender:\ TObject); \\
\mbox{}\textbf{var} \\
\mbox{}\ \ i,\ j\ :\ integer; \\
\mbox{}\ \ ColorT\ :\ TColor; \\
\mbox{}\ \ N\ :\ integer; \\
\mbox{}\textbf{begin} \\
\mbox{}\ \ FieldPaintBox.Width:=CellSize*ChessSize; \\
\mbox{}\ \ FieldPaintBox.Height:=CellSize*ChessSize; \\
\mbox{} \\
\mbox{}\ \ Width:=200+CellSize*ChessSize; \\
\mbox{}\ \ Height:=CellSize*ChessSize; \\
\mbox{} \\
\mbox{}\ \ \textit{//\ Рисуем\ поле} \\
\mbox{}\ \ \textbf{for}\ i\ :=\ 0\ \textbf{to}\ ChessSize-1\ \textbf{do} \\
\mbox{}\ \ \ \ \textbf{for}\ j\ :=\ 0\ \textbf{to}\ ChessSize-1\ \textbf{do} \\
\mbox{}\ \ \ \ \ \ \textbf{with}\ FieldPaintBox.Canvas\ \textbf{do}\ \textbf{begin} \\
\mbox{}\ \ \ \ \ \ \ \ \textit{//\ Выберем\ цвет\ клетки} \\
\mbox{}\ \ \ \ \ \ \ \ \textbf{if}\ ((ChessSize\ -\ i)\ +\ j)\ \textbf{mod}\ 2\ =\ 1\ \textbf{then} \\
\mbox{}\ \ \ \ \ \ \ \ \ \ ColorT\ :=\ \textbf{RGB}(70,35,0)\ \textit{//\ Темно-коричневый} \\
\mbox{}\ \ \ \ \ \ \ \ \textbf{else} \\
\mbox{}\ \ \ \ \ \ \ \ \ \ ColorT\ :=\ \textbf{RGB}(240,\ 240,\ 220);\ \textit{//\ Слоновая\ кость} \\
\mbox{}\ \ \ \ \ \ \ \ Pen.Color\ :=\ ColorT; \\
\mbox{}\ \ \ \ \ \ \ \ Brush.Color\ :=\ ColorT; \\
\mbox{}\ \ \ \ \ \ \ \ \textbf{FillRect}(i*CellSize,\ j*CellSize,\ (i+1)*CellSize,\ (j+1)*CellSize); \\
\mbox{}\ \ \ \ \ \ \textbf{end}; \\
\mbox{} \\
\mbox{}\ \ \textbf{with}\ FieldPaintBox.Canvas\ \textbf{do}\ \textbf{begin} \\
\mbox{}\ \ \ \ \textit{//\ Рамочка\ вокруг\ поля} \\
\mbox{}\ \ \ \ Pen.Color\ :=\ clWhite; \\
\mbox{}\ \ \ \ Brush.Color\ :=\ clBlack; \\
\mbox{}\ \ \ \ \textbf{FrameRect}(0,0,CellSize*ChessSize,CellSize*ChessSize); \\
\mbox{}\ \ \ \ \textit{//\ Рисуем\ коня} \\
\mbox{}\ \ \ \ \textbf{Draw}(HX*CellSize+(CellSize-Horse.Bitmap.Width)\ \textbf{div}\ 2, \\
\mbox{}\ \ \ \ \ \ \ \ \ \ \ \ \ \ \ \ \ \ \ \ \ \ HY*CellSize,Horse.Bitmap); \\
\mbox{}\ \ \ \ \textit{//\ Рисуем\ траву} \\
\mbox{}\ \ \ \ \textbf{Draw}(GX*CellSize+(CellSize-Grass.Bitmap.Width)\ \textbf{div}\ 2, \\
\mbox{}\ \ \ \ \ \ \ \ \ \ \ \ \ \ \ \ \ \ \ \ \ \ GY*CellSize,Grass.Bitmap); \\
\mbox{} \\
\mbox{}\ \ \ \ \textit{//\ Отображение\ пути} \\
\mbox{}\ \ \ \ N\ :=\ HX\ +\ HY*ChessSize; \\
\mbox{}\ \ \ \ i\ :=\ -1; \\
\mbox{}\ \ \ \ \textbf{repeat} \\
\mbox{}\ \ \ \ \ \ \textbf{inc}(i); \\
\mbox{}\ \ \ \ \ \ \textbf{Draw}((Path[i]\ \textbf{mod}\ ChessSize)*CellSize, \\
\mbox{}\ \ \ \ \ \ \ \ \ \ \ (Path[i]\ \textbf{div}\ ChessSize)*CellSize, \\
\mbox{}\ \ \ \ \ \ \ \ \ \ \ \ Check.Bitmap); \\
\mbox{}\ \ \ \ \textbf{until}\ Path[i]\ =\ N; \\
\mbox{} \\
\mbox{}\ \ \ \ \textit{//\ Прорисуем\ доступные\ для\ прыжка\ клетки} \\
\mbox{}\ \ \ \ \textit{//\ Нужно\ же\ как-то\ проверить\ алгоритм\ генерации} \\
\mbox{}\ \ \ \ \textbf{if}\ (ShowDistCheckBox.Checked)\ \textbf{then}\ \textbf{begin} \\
\mbox{}\ \ \ \ \ \ Pen.Color\ :=\ clBlack; \\
\mbox{}\ \ \ \ \ \ Brush.Color:=clWhite; \\
\mbox{}\ \ \ \ \ \ \textbf{for}\ i\ :=\ 0\ \textbf{to}\ ChessSize-1\ \textbf{do} \\
\mbox{}\ \ \ \ \ \ \ \ \textbf{for}\ j\ :=\ 0\ \textbf{to}\ ChessSize-1\ \textbf{do} \\
\mbox{}\ \ \ \ \ \ \ \ \ \ \textbf{TextOut}(i*CellSize,\ j*CellSize,\ \textbf{IntToStr}(M[i][j])); \\
\mbox{}\ \ \ \ \textbf{end}; \\
\mbox{}\ \ \textbf{end}; \\
\mbox{}\ \ HorseCoordLabel.Caption\ :=\ \texttt{'Координаты\ коня:\ '}\ +\ \textbf{IntToStr}(HX+1) \\
\mbox{}\ \ \ \ \ \ \ \ \ \ \ \ \ \ \ \ \ \ \ \ \ \ \ \ \ \ \ \ \ \ \ \ \ \ \ \ \ \ \ \ \ \ \ \ \ \ \ \ \ +\ \texttt{','}\ +\ \textbf{IntToStr}(HY+1); \\
\mbox{}\ \ GrassCoordLabel.Caption\ :=\ \texttt{'Координаты\ травы:\ '}\ +\ \textbf{IntToStr}(GX+1) \\
\mbox{}\ \ \ \ \ \ \ \ \ \ \ \ \ \ \ \ \ \ \ \ \ \ \ \ \ \ \ \ \ \ \ \ \ \ \ \ \ \ \ \ \ \ \ \ \ \ \ \ \ \ +\ \texttt{','}\ +\ \textbf{IntToStr}(GY+1); \\
\mbox{}\textbf{end}; \\
\mbox{} \\
\mbox{}\textbf{procedure}\ TMainForm.\textbf{ApplySizeButtonClick}(Sender:\ TObject); \\
\mbox{}\textbf{begin} \\
\mbox{}\ \ \textit{//\ Изменим\ размеры\ поля} \\
\mbox{}\ \ ChessSize\ :=\ SizeSpinEdit.Value; \\
\mbox{}\ \ \textit{//\ Подвинем\ коня\ и\ траву} \\
\mbox{}\ \ \textbf{if}\ (HX\ \textgreater{}=\ ChessSize)\ \textbf{or}\ (HY\ \textgreater{}=\ ChessSize)\ \textbf{then}\ \textbf{begin} \\
\mbox{}\ \ \ \ HX\ :=\ 0; \\
\mbox{}\ \ \ \ HY\ :=\ 0; \\
\mbox{}\ \ \textbf{end}; \\
\mbox{}\ \ \textbf{if}\ (GX\ \textgreater{}=\ ChessSize)\ \textbf{or}\ (GY\ \textgreater{}=\ ChessSize)\ \textbf{then}\ \textbf{begin} \\
\mbox{}\ \ \ \ GX\ :=\ ChessSize-1; \\
\mbox{}\ \ \ \ GY\ :=\ ChessSize-1; \\
\mbox{}\ \ \textbf{end}; \\
\mbox{}\ \ CalcPath; \\
\mbox{}\ \ Repaint; \\
\mbox{}\textbf{end}; \\
\mbox{} \\
\mbox{}\textbf{procedure}\ TMainForm.\textbf{AboutItemClick}(Sender:\ TObject); \\
\mbox{}\textbf{begin} \\
\mbox{}\ \ AboutBox.ShowModal; \\
\mbox{}\textbf{end}; \\
\mbox{} \\
\mbox{}\textbf{procedure}\ TMainForm.\textbf{ExitItemClick}(Sender:\ TObject); \\
\mbox{}\textbf{begin} \\
\mbox{}\ \ Close; \\
\mbox{}\textbf{end}; \\
\mbox{} \\
\mbox{}\textbf{procedure}\ TMainForm.CalcPath; \\
\mbox{}\textbf{var} \\
\mbox{}\ \ N\ :\ integer; \\
\mbox{}\textbf{begin} \\
\mbox{}\ \ N\ :=\ GetTickCount; \\
\mbox{}\ \ \textit{//\ Рассчет\ пути\ коня} \\
\mbox{}\ \ \textbf{BuildHorseMatrix2}(M,\ ChessSize,\ HX,\ HY,\ GX,\ GY); \\
\mbox{}\ \ \textbf{ReverseWave}(M,\ ChessSize,\ HX,\ HY,\ GX,\ GY,\ Path); \\
\mbox{}\ \ \textit{//DumpTo(Path,\ ChessSize,\ 'Path.txt');} \\
\mbox{}\ \ N\ :=\ GetTickCount\ -\ N; \\
\mbox{}\ \ MovesLabel.Caption\ :=\ \texttt{'Ходов\ коня:\ '}\ +\ \textbf{IntToStr}(M[GX][GY]); \\
\mbox{}\ \ TimeLabel.Caption\ :=\ \textbf{IntToStr}(N)\ +\ \texttt{'\ миллисекунд'}; \\
\mbox{}\textbf{end}; \\
\mbox{} \\
\mbox{}\textbf{procedure}\ TMainForm.\textbf{ShowDistCheckBoxChange}(Sender:\ TObject); \\
\mbox{}\textbf{begin} \\
\mbox{}\ \ Repaint; \\
\mbox{}\textbf{end}; \\
\mbox{} \\
\mbox{}\textbf{procedure}\ TMainForm.\textbf{FieldPaintBoxMouseDown}(Sender:\ TObject; \\
\mbox{}\ \ Button:\ TMouseButton;\ Shift:\ TShiftState;\ X,\ Y:\ Integer); \\
\mbox{}\textbf{var} \\
\mbox{}\ \ NX,\ NY\ :\ integer; \\
\mbox{}\textbf{begin} \\
\mbox{}\ \ \textbf{if}\ (ssLeft\ \textbf{in}\ Shift)\ \textbf{then}\ \textbf{begin} \\
\mbox{}\ \ \ \ NX\ :=\ X\ \textbf{div}\ CellSize; \\
\mbox{}\ \ \ \ NY\ :=\ Y\ \textbf{div}\ CellSize; \\
\mbox{}\ \ \ \ \textbf{if}\ ((NX\ \textless{}\ ChessSize)\ \textbf{and}\ (NY\ \textless{}\ ChessSize))\ \textbf{then}\ \textbf{begin} \\
\mbox{}\ \ \ \ \ \ HX\ :=\ NX; \\
\mbox{}\ \ \ \ \ \ HY\ :=\ NY; \\
\mbox{}\ \ \ \ \textbf{end}; \\
\mbox{}\ \ \textbf{end}\ \textbf{else}\ \textbf{if}\ (ssRight\ \textbf{in}\ Shift)\ \textbf{then}\ \textbf{begin} \\
\mbox{}\ \ \ \ NX\ :=\ X\ \textbf{div}\ CellSize; \\
\mbox{}\ \ \ \ NY\ :=\ Y\ \textbf{div}\ CellSize; \\
\mbox{}\ \ \ \ \textbf{if}\ ((NX\ \textless{}\ ChessSize)\ \textbf{and}\ (NY\ \textless{}\ ChessSize))\ \textbf{then}\ \textbf{begin} \\
\mbox{}\ \ \ \ \ \ GX\ :=\ NX; \\
\mbox{}\ \ \ \ \ \ GY\ :=\ NY; \\
\mbox{}\ \ \ \ \textbf{end}; \\
\mbox{}\ \ \textbf{end}; \\
\mbox{}\ \ CalcPath; \\
\mbox{}\ \ Repaint; \\
\mbox{}\textbf{end}; \\
\mbox{} \\
\mbox{}\textbf{procedure}\ TMainForm.\textbf{FormClose}(Sender:\ TObject;\ \textbf{var}\ CloseAction:\ TCloseAction); \\
\mbox{}\textbf{var} \\
\mbox{}\ \ f\ :\ \textbf{text}; \\
\mbox{}\ \ i,N\ :\ integer; \\
\mbox{}\textbf{begin} \\
\mbox{}\ \ Horse.Free; \\
\mbox{}\ \ Grass.Free; \\
\mbox{}\ \ Check.Free; \\
\mbox{}\ \ \textbf{assignFile}(f,\ OUTFILE); \\
\mbox{}\ \ \textbf{rewrite}(f); \\
\mbox{}\ \ \textbf{writeln}(F,\ M[GX][GY]); \\
\mbox{}\ \ N\ :=\ -1; \\
\mbox{}\ \ \textbf{repeat} \\
\mbox{}\ \ \ \ \textbf{inc}(N); \\
\mbox{}\ \ \textbf{until}\ (Path[N]\ =\ HX\ +\ HY*ChessSize); \\
\mbox{} \\
\mbox{}\ \ \textbf{for}\ i\ :=\ N\ \textbf{downto}\ 0\ \textbf{do} \\
\mbox{}\ \ \ \ \textbf{writeln}(f,\ (Path[i]\ \textbf{mod}\ ChessSize)\ +\ 1,\ \texttt{'\ '},\ (Path[i]\ \textbf{div}\ ChessSize)\ +\ 1); \\
\mbox{} \\
\mbox{}\ \ \textbf{closeFile}(f); \\
\mbox{}\textbf{end}; \\
\mbox{} \\
\mbox{}\textbf{end}. \\
\mbox{} \\
\mbox{}
\\
\fbox{Модуль GraphWork:}\\
% Generator: GNU source-highlight, by Lorenzo Bettini, http://www.gnu.org/software/src-highlite
\noindent
\mbox{}unit\ GraphWork; \\
\mbox{} \\
\mbox{}\textit{\{\$mode\ objfpc\}\{\$H+\}} \\
\mbox{} \\
\mbox{}interface \\
\mbox{} \\
\mbox{}\textbf{uses} \\
\mbox{}\ \ Classes,\ SysUtils; \\
\mbox{} \\
\mbox{} \\
\mbox{}\textbf{type} \\
\mbox{}\ \ \textit{//\ Матрица\ смежности} \\
\mbox{}\ \ Graph\ =\ \textbf{array}\ \textbf{of}\ \textbf{array}\ \textbf{of}\ integer; \\
\mbox{}\ \ IArray\ =\ \textbf{array}\ \textbf{of}\ integer; \\
\mbox{} \\
\mbox{}\textbf{procedure}\ \textbf{DumpTo}(\textbf{var}\ M\ :\ Graph;\ N\ :\ integer;\ FName\ :\ string); \\
\mbox{}\textbf{procedure}\ \textbf{BuildHorseMatrix2}(\textbf{var}\ M\ :\ Graph;\ N\ :\ integer; \\
\mbox{}\ \ \ \ \ \ \ \ \ \ \ \ \ \ \ \ \ \ \ \ \ \ \ \ \ \ \ \ X1,\ Y1,\ X2,\ Y2\ :\ integer); \\
\mbox{}\textbf{procedure}\ \textbf{ReverseWave}(\textbf{var}\ M\ :\ Graph;\ N\ :\ integer; \\
\mbox{}\ \ \ \ \ \ \ \ \ \ \ \ \ \ \ \ \ \ \ \ \ \ X1,\ Y1,\ X2,\ Y2\ :\ integer; \\
\mbox{}\ \ \ \ \ \ \ \ \ \ \ \ \ \ \ \ \ \ \ \ \ \ \textbf{var}\ Path\ :\ IArray); \\
\mbox{} \\
\mbox{}implementation \\
\mbox{} \\
\mbox{}\textbf{procedure}\ \textbf{Mark}(\textbf{var}\ M\ :\ Graph;\ N,d\ :\ integer; \\
\mbox{}\ \ \ \ \ \ \ \ \ \ \ \ \ \ \ x,y\ :\ integer; \\
\mbox{}\ \ \ \ \ \ \ \ \ \ \ \ \ \ \ X1,Y1\ :\ integer); \\
\mbox{}\textbf{begin} \\
\mbox{}\ \ \textbf{if}\ (X\ \textgreater{}=\ 0)\ \textbf{AND}\ (X\ \textless{}\ N)\ \textbf{AND} \\
\mbox{}\ \ \ \ \ (Y\ \textgreater{}=\ 0)\ \textbf{AND}\ (Y\ \textless{}\ N)\ \textbf{AND} \\
\mbox{}\ \ \ \ \ (M[X][Y]\ \textgreater{}\ d)\ \textbf{then}\ \textbf{begin} \\
\mbox{} \\
\mbox{}\ \ \ \ M[x][y]\ :=\ d; \\
\mbox{} \\
\mbox{}\ \ \ \ \textbf{Mark}(M,\ N,\ d+1,\ x+1,\ y+2,\ X1,\ Y1); \\
\mbox{}\ \ \ \ \textbf{Mark}(M,\ N,\ d+1,\ x+1,\ y-2,\ X1,\ Y1); \\
\mbox{}\ \ \ \ \textbf{Mark}(M,\ N,\ d+1,\ x-1,\ y+2,\ X1,\ Y1); \\
\mbox{}\ \ \ \ \textbf{Mark}(M,\ N,\ d+1,\ x-1,\ y-2,\ X1,\ Y1); \\
\mbox{}\ \ \ \ \textbf{Mark}(M,\ N,\ d+1,\ x+2,\ y+1,\ X1,\ Y1); \\
\mbox{}\ \ \ \ \textbf{Mark}(M,\ N,\ d+1,\ x+2,\ y-1,\ X1,\ Y1); \\
\mbox{}\ \ \ \ \textbf{Mark}(M,\ N,\ d+1,\ x-2,\ y+1,\ X1,\ Y1); \\
\mbox{}\ \ \ \ \textbf{Mark}(M,\ N,\ d+1,\ x-2,\ y-1,\ X1,\ Y1); \\
\mbox{}\ \ \textbf{end}; \\
\mbox{}\textbf{end}; \\
\mbox{} \\
\mbox{}\textit{//\ Прямой\ ход\ алгоритма\ Ли\ -\ размечаем\ матрицу\ расстояний} \\
\mbox{}\textbf{procedure}\ \textbf{BuildHorseMatrix2}(\textbf{var}\ M\ :\ Graph;\ N\ :\ integer; \\
\mbox{}\ \ \ \ \ \ \ \ \ \ \ \ \ \ \ \ \ \ \ \ \ \ \ \ \ \ \ \ X1,\ Y1,\ X2,\ Y2\ :\ integer); \\
\mbox{}\textbf{var} \\
\mbox{}\ \ i,j\ :\ integer; \\
\mbox{}\textbf{begin} \\
\mbox{}\ \ \textbf{SetLength}(M,\ N,\ N); \\
\mbox{}\ \ \textbf{for}\ i\ :=\ 0\ \textbf{to}\ N-1\ \textbf{do} \\
\mbox{}\ \ \ \ \textbf{for}\ j\ :=\ 0\ \textbf{to}\ N-1\ \textbf{do} \\
\mbox{}\ \ \ \ \ \ M[i][j]\ :=\ N*N; \\
\mbox{}\ \ \textbf{Mark}(M,\ N,\ 0,\ X1,\ Y1,\ X2,\ Y2); \\
\mbox{}\textbf{end}; \\
\mbox{} \\
\mbox{}\textit{//\ Поиск\ соседней\ ячейки} \\
\mbox{}\textit{//\ Ячейка\ считается\ соседней,\ если\ она\ доступна\ за\ один\ шаг} \\
\mbox{}\textit{//\ и\ значение\ в\ ней\ меньше\ на\ единичку} \\
\mbox{}\textbf{function}\ \textbf{FindNear}(\textbf{var}\ M\ :\ Graph;\ N\ :\ integer; \\
\mbox{}\ \ \ \ \ \ \ \ \ \ \ \ \ \ \ \ \ \ X,\ Y\ :\ integer)\ :\ integer; \\
\mbox{}\textbf{var} \\
\mbox{}\ \ d\ :\ integer; \\
\mbox{}\textbf{begin} \\
\mbox{}\ \ D\ :=\ M[X][Y]; \\
\mbox{} \\
\mbox{}\ \ \textbf{if}\ (X+1\ \textless{}\ N)\ \textbf{AND}\ (Y+2\ \textless{}\ N)\ \textbf{AND}\ (M[X+1][Y+2]\ \textless{}\ d)\ \textbf{then} \\
\mbox{}\ \ \ \ FindNear\ :=\ X+1\ +\ (Y+2)*N \\
\mbox{}\ \ \textbf{else}\ \textbf{if}\ (X+1\ \textless{}\ N)\ \textbf{AND}\ (Y-2\ \textgreater{}=\ 0)\ \textbf{AND}\ (M[X+1][Y-2]\ \textless{}\ d)\ \textbf{then} \\
\mbox{}\ \ \ \ FindNear\ :=\ X+1\ +\ (Y-2)*N \\
\mbox{}\ \ \textbf{else}\ \textbf{if}\ (X-1\ \textgreater{}=\ 0)\ \textbf{AND}\ (Y+2\ \textless{}\ N)\ \textbf{AND}\ (M[X-1][Y+2]\ \textless{}\ d)\ \textbf{then} \\
\mbox{}\ \ \ \ FindNear\ :=\ X-1\ +\ (Y+2)*N \\
\mbox{}\ \ \textbf{else}\ \textbf{if}\ (X-1\ \textgreater{}=\ 0)\ \textbf{AND}\ (Y-2\ \textgreater{}=\ 0)\ \textbf{AND}\ (M[X-1][Y-2]\ \textless{}\ d)\ \textbf{then} \\
\mbox{}\ \ \ \ FindNear\ :=\ X-1\ +\ (Y-2)*N \\
\mbox{}\ \ \textbf{else}\ \textbf{if}\ (X+2\ \textless{}\ N)\ \textbf{AND}\ (Y+1\ \textless{}\ N)\ \textbf{AND}\ (M[X+2][Y+1]\ \textless{}\ d)\ \textbf{then} \\
\mbox{}\ \ \ \ FindNear\ :=\ X+2\ +\ (Y+1)*N \\
\mbox{}\ \ \textbf{else}\ \textbf{if}\ (X+2\ \textless{}\ N)\ \textbf{AND}\ (Y-1\ \textgreater{}=\ 0)\ \textbf{AND}\ (M[X+2][Y-1]\ \textless{}\ d)\ \textbf{then} \\
\mbox{}\ \ \ \ FindNear\ :=\ X+2\ +\ (Y-1)*N \\
\mbox{}\ \ \textbf{else}\ \textbf{if}\ (X-2\ \textgreater{}=\ 0)\ \textbf{AND}\ (Y+1\ \textless{}\ N)\ \textbf{AND}\ (M[X-2][Y+1]\ \textless{}\ d)\ \textbf{then} \\
\mbox{}\ \ \ \ FindNear\ :=\ X-2\ +\ (Y+1)*N \\
\mbox{}\ \ \textbf{else}\ \textbf{if}\ (X-2\ \textgreater{}=\ 0)\ \textbf{AND}\ (Y-1\ \textgreater{}=\ 0)\ \textbf{AND}\ (M[X-2][Y-1]\ \textless{}\ d)\ \textbf{then} \\
\mbox{}\ \ \ \ FindNear\ :=\ X-2\ +\ (Y-1)*N \\
\mbox{}\ \ \textbf{else}\ \textbf{begin} \\
\mbox{}\ \ \ \ raise\ Exception.\textbf{Create}(\texttt{'Something\ wrong!'}); \\
\mbox{}\ \ \ \ FindNear\ :=\ -1; \\
\mbox{}\ \ \textbf{end}; \\
\mbox{}\textbf{end}; \\
\mbox{} \\
\mbox{}\textit{//\ Обратный\ ход\ алгоритма\ Ли} \\
\mbox{}\textbf{procedure}\ \textbf{ReverseWave}(\textbf{var}\ M\ :\ Graph;\ N\ :\ integer; \\
\mbox{}\ \ \ \ \ \ \ \ \ \ \ \ \ \ \ \ \ \ \ \ \ \ X1,\ Y1,\ X2,\ Y2\ :\ integer; \\
\mbox{}\ \ \ \ \ \ \ \ \ \ \ \ \ \ \ \ \ \ \ \ \ \ \textbf{var}\ Path\ :\ IArray); \\
\mbox{}\textbf{var} \\
\mbox{}\ \ i\ :\ integer; \\
\mbox{}\textbf{begin} \\
\mbox{}\ \ \textbf{SetLength}(Path,\ N*N); \\
\mbox{}\ \ i\ :=\ 0; \\
\mbox{}\ \ Path[0]\ :=\ X2\ +\ Y2*N; \\
\mbox{}\ \ \textbf{while}\ ((X1\ \textless{}\textgreater{}\ X2)\ \textbf{or}\ (Y1\ \textless{}\textgreater{}\ Y2))\ \textbf{do}\ \textbf{begin} \\
\mbox{}\ \ \ \ \textbf{inc}(i); \\
\mbox{}\ \ \ \ Path[i]\ :=\ \textbf{FindNear}(M,\ N,\ X2,\ Y2); \\
\mbox{}\ \ \ \ X2\ :=\ Path[i]\ \textbf{mod}\ N; \\
\mbox{}\ \ \ \ Y2\ :=\ Path[i]\ \textbf{div}\ N; \\
\mbox{}\ \ \textbf{end}; \\
\mbox{}\ \ Path[i+1]\ :=\ X1\ +\ Y1*N; \\
\mbox{}\textbf{end}; \\
\mbox{} \\
\mbox{}\textbf{procedure}\ \textbf{DumpTo}(\textbf{var}\ M\ :\ Graph;\ N\ :\ integer;\ FName\ :\ string); \\
\mbox{}\textbf{var} \\
\mbox{}\ \ F\ :\ \textbf{text}; \\
\mbox{}\ \ i,j\ :\ integer; \\
\mbox{}\textbf{begin} \\
\mbox{}\ \ \textbf{assignFile}(f,\ fname); \\
\mbox{}\ \ \textbf{rewrite}(f); \\
\mbox{} \\
\mbox{}\ \ \textbf{for}\ i\ :=\ 0\ \textbf{to}\ N-1\ \textbf{do}\ \textbf{begin} \\
\mbox{}\ \ \ \ \textbf{for}\ j\ :=\ 0\ \textbf{to}\ N-1\ \textbf{do} \\
\mbox{}\ \ \ \ \ \ \textbf{write}(f,\ M[i][j],\ \texttt{'\ '}); \\
\mbox{}\ \ \ \ \textbf{writeln}(f); \\
\mbox{}\ \ \textbf{end}; \\
\mbox{} \\
\mbox{}\ \ \textbf{closeFile}(f); \\
\mbox{}\textbf{end}; \\
\mbox{} \\
\mbox{}\textbf{end}. \\
\mbox{} \\
\mbox{}

\fbox{Модуль AboutBox:}\\
% Generator: GNU source-highlight, by Lorenzo Bettini, http://www.gnu.org/software/src-highlite
\noindent
\mbox{}unit\ Unit2; \\
\mbox{} \\
\mbox{}\textit{\{\$mode\ objfpc\}\{\$H+\}} \\
\mbox{} \\
\mbox{}interface \\
\mbox{} \\
\mbox{}\textbf{uses} \\
\mbox{}\ \ Classes,\ SysUtils,\ FileUtil,\ Forms,\ Controls,\ Graphics,\ Dialogs, \\
\mbox{}\ \ StdCtrls,\ LCLType,\ ExtCtrls; \\
\mbox{} \\
\mbox{}\textbf{type} \\
\mbox{} \\
\mbox{}\ \ \textit{\{\ TAboutBox\ \}} \\
\mbox{} \\
\mbox{}\ \ TAboutBox\ =\ \textbf{class}(TForm) \\
\mbox{}\ \ \ \ AboutBoxLabel:\ TLabel; \\
\mbox{}\ \ \ \ Button1:\ TButton; \\
\mbox{}\ \ \ \ Image1:\ TImage; \\
\mbox{}\ \ \ \ Memo1:\ TMemo; \\
\mbox{}\ \ \ \ \textbf{procedure}\ \textbf{Button1Click}(Sender:\ TObject); \\
\mbox{}\ \ private \\
\mbox{}\ \ \ \ \textit{\{\ private\ declarations\ \}} \\
\mbox{}\ \ public \\
\mbox{}\ \ \ \ \textit{\{\ public\ declarations\ \}} \\
\mbox{}\ \ \textbf{end}; \\
\mbox{} \\
\mbox{}\textbf{var} \\
\mbox{}\ \ AboutBox:\ TAboutBox; \\
\mbox{} \\
\mbox{}\textbf{procedure}\ \textbf{ShowError}(E\ :\ PChar); \\
\mbox{} \\
\mbox{}implementation \\
\mbox{} \\
\mbox{}\textit{\{\$R\ *.lfm\}} \\
\mbox{} \\
\mbox{}\textbf{procedure}\ \textbf{ShowError}(E\ :\ PChar); \\
\mbox{}\textbf{begin} \\
\mbox{}\ \ Application.\textbf{MessageBox}(E, \\
\mbox{}\ \ \ \ \ \ \ \ \ \ \ \ \ \ \ \ \ \texttt{'Произошла\ ошибка'}, \\
\mbox{}\ \ \ \ \ \ \ \ \ \ \ \ \ \ \ \ \ MB$\_$OK\ \textbf{OR}\ MB$\_$ICONERROR) \\
\mbox{}\textbf{end}; \\
\mbox{} \\
\mbox{}\textit{\{\ TAboutBox\ \}} \\
\mbox{} \\
\mbox{}\textbf{procedure}\ TAboutBox.\textbf{Button1Click}(Sender:\ TObject); \\
\mbox{}\textbf{begin} \\
\mbox{}\ \ Close; \\
\mbox{}\textbf{end}; \\
\mbox{} \\
\mbox{}\textbf{end}. \\
\mbox{} \\
\mbox{}

\pagebreak

\begin{thebibliography}{99}
\bibitem{dijkstra}
  Дейкстра Э.В.,
  \emph{Дисциплина прграммирования},
  Москва, Мир,
  1978
\bibitem{plaksin}
  Плаксин М.А.,
  \emph{Тестирование и отладка},
  Москва, Бином,
  2009
\bibitem{korolev}
  Королев Л.Н, Миков А.И.,
  \emph{Информатика. Введение в компьютерные науки},
  Москва, Арбис,
  2014
\bibitem{wiki}
  Wikipedia,
  \emph{http://ru.wikipedia.org/wiki/Алгоритм\_Ли}
\end{thebibliography}
\pagebreak

\end{document}
