\section{Детальное описание алгоритма}
\DeclareRobustCommand{\l}[1]{
\item {[}{\it #1}{]} 
}
Алгоритм состоит из следующих основных процедур:
\begin{itemize}
\item main - главная процедура, вызывает MainLoop для организации диалога с
пользователем
\item MainLoop - организует приглашение командной строки, по запросу пользователя
запускает задачу, показывает справку или завершает работу программы
\item ShowHelp - демонстрирует экран помощи
\item RunFile - запрашивает у пользователя имя файла и в случае его
корректности запускает задачу - ReadFile
\item ReadFile - основная задача, считывает из файла построчно выражения и
передает их процедуре Evaluate
\item Evaluate - производит действия над выражениями, используя
CreateInfixTree, Simplify и DumpTree
\item CreateInfixTree - выполняет разбор записанного в строке выражения во внутренний формат
программы (дерево), используя алгоритм Дейкстры
\item Build - создает один узел будущего дерева
\item CopyTree - создает копию дерева
\item Simplify - выполняет упрощение (на самом деле, просто преобразование)
дерева согласно указанным правилам 
\item Calculate - вычисляет значение выражения
\end{itemize}

Детальное описание:
\begin{enumerate}
  \l{main} Исполнение программы начинается с функции main
  \l{MainLoop} Выводим на экран приветствие
  \l{MainLoop} Запускаем главный цикл:
  \begin{enumerate}
    \l{MainLoop} Считываем ввод пользователя, в соответствии с этим переходим
    или к RunString, или к RunFile
    \begin{enumerate}
      \l{RunFile} Открываем файл, считываем каждую строку и передаем Evaluate
      \begin{enumerate}
      \l{Evaluate} Строим дерево
      \l{CreateInfixTree} Разбор происходит итеративным способом с
      использованием стека;
      первоначально выражение преобразуется в обратную польскую запись, после
      по ней строится дерево
      \l{Evaluate} Если преобразование прошло успешно, пробуем упростить
      выражение
      \l{Simplify} Начиная с листьев, поднимаемся вверх и по пути выполняем
      необходимые преобразования
      \l{Evaluate} Выводим полученное дерево на экран
      \l{Calculate} Если не произошло неприятностей, вычисляем значение
      выражения
      \end{enumerate}
      \l{RunString} Запрашиваем у пользователя строку и, аналогично, передаем
      ее Evaluate
    \end{enumerate}
    \l{MainLoop} Запрашиваем у пользователя подтверждение на продолжение
    сеанса взаимодействия с программой
    \l{MainLoop} Если пользователь ответил отрицательно, то выходим из
    цикла
  \end{enumerate}
  \l{main} Завершение работы программы
\end{enumerate}

\pagebreak

