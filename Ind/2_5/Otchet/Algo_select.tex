\section{Определение идеи алгоритма, выбор методов решения и структур данных}
\par
Перед программой стоит задача - считать из файла выражение, преобразовать его
в виде дерева, упростить его и вывести в терминал.
\par
Для начала определимся с типами данных. По условию задачи необходимо
использовать в качестве внутреннего представления дерево. Определим тип запись
с полями, равными типу элемента (константа, переменная, оператор), значению
элемента (код символа - для операции и переменной и просто число - для целого
числа). Также структура содержит три поля ссылочного типа - два для
формирования дерева и одно - для формирования обратной польской записи.
\par
Ввод данных удобно организовать через файл; также стоит предусмотреть
возможность ввода строки с клавиатуры для удобства вычисления одиночного
выражения.
\par
Программа будет состоять из трех модулей - модуль ввода данных, модуль их
преобразования и модуль вывода. В модуле ввода центральное место займет
анализатор выражений, построенный по принципу сортировочной станции; в
модуле вывода - симметричный ему алгоритм
конвертации внутреннего представления в строку. В модуле преобразования
будет рекурсивная процедура, выполняющая указанные в задаче упрощения.
\par
В качестве отдельного Unit-файла я выделил операции работы со стеком и
подпрограмму обработки ошибок, что позволило удобно использовать эти процедуры
в основной программе.
\par
Пару слов насчет интерфейса. Поскольку основной ввод/вывод организуется через
файл, у пользователя необходимо запросить только лишь его имя (и проверить на
существование). Но в случае ошибки было бы неплохо указать, где именно
споткнулась программа (а такое место может быть только одно - анализатор
выражений), и в любом случае - вывести какое-нибудь сообщение по окончании
работы программы (``Все хорошо, хозяин'' или ``Явка провалена из-за ошибки
резидента'').
\pagebreak

