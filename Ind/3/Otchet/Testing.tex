\section{Тестирование программы}
\newcounter{testcnt}
\DeclareRobustCommand{\test}[3]{
  \addtocounter{testcnt}{1}
  \par
  Тест \arabic{testcnt}: {#1}\\
  {\it Входные данные:} {#2}\\
  {\it Выходные данные:} {#3}\\
  {\it Ожидаемый результат:} {#3}\\
  \begin {figure}[H] 
  \centering 
  \includegraphics[scale=0.5]{test\arabic{testcnt}.png} 
  \caption{Скриншот} 
  \label{fig:scr\arabic{testcnt}} 
  \end {figure}
}
\par
Тесты для программы были разбиты на несколько групп таким образом, чтобы
покрыть одновременно максимальное количество типов входных данных и вариантов
выполнения тела программы.\\
%\test{}{}{}{}
\test{Результат выполнения программы с нормальными входными данными}{Смотри
рисунок}{Корректные выходные данные}
\test{Выполнение программы с корректными входными данными}{Смотри предыдущий
рисунок}{Правильное отображение элементов}
\test{Обработка ситуации отсутствия входного файла}{Входной файл
отсутствует}{Всем параметрам присваивается значение по умолчанию}
\test{Некорректные входные данные}{Файл с некорректными данными}{Всем
параметрам присваиваются значения по умолчанию}
\test{Перемещение элементов}{Пользователь кликом мыши перемещает коня и
траву}{Конь и трава корректно перемещаются, путь пересчитывается}
\test{Динамическое изменение размера поля на больший}{Пользователь задает новый размер
поля}{Поле меняет размер, конь и трава не перемещаются; возможно,
пересчитывается путь}
\test{Динамическое изменение размера поля на меньший}{Пользователь задает
новый размер поля, конь и/или трава в него не помещаются}{Конь и/или трава
меняют свое положение на допустимые}
\pagebreak
