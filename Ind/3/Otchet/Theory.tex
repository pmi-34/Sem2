\section{Теория}
\par
Прежде чем приступать к реализации задания, будет хорошей идеей
ознакомиться с имеющейся теорией по данному вопросу.\\
В качестве алгоритма поиска кратчайшего пути вначале был выбран алгоритм
Дейкстры. Программа работала корректно, но для больших размерностей доски
($20\times{20}$, например) заметно задумывалась - порой на десятки секунд.
Поэтому для конечной реализации программы алгоритм Дейкстры был заменен на
алгоритм волновой трассировки. Это позволило существенно ускорить работу программы,
переведя ее в режим реального времени.
{\bf Алгоритм волновой трассировки} -алгоритм поиска кратчайшего пути в графе. 
Принадлежит к алгоритмам, основанным на методах поиска в ширину.\\
В основном используется при компьютерной трассировке (разводке) печатных плат,
соединительных проводников на поверхности микросхем. Другое применение
волнового алгоритма - поиск кратчайшего расстояния на карте в компьютерных
стратегических играх.\\
Алгоритм работает на дискретном рабочем поле, представляющем собой
ограниченную замкнутой линией фигуру, не обязательно прямоугольную, разбитую
на прямоугольные ячейки, в частном случае - квадратные. Множество всех ячеек
поля разбивается на четыре подмножества: 
\begin{enumerate}
\item ``проходимые'' (свободные), то есть 
при поиске пути их можно проходить 
\item ``непроходимые'' (препятствия), путь через 
эту ячейку запрещён 
\item стартовая ячейка (источник)
\item финишная (приемник)
\end{enumerate}
Назначение
стартовой и финишной ячеек условно, достаточно указать пару ячеек, между
которыми нужно найти кратчайший путь.\\
Работа алгоритма включает в себя три этапа: инициализацию, распространение
волны и восстановление пути.\\
Во время инициализации строится образ множества ячеек обрабатываемого поля,
каждой ячейке приписываются атрибуты проходимости/непроходимости, запоминаются
стартовая и финишная ячейки. После инициализации алгоритм уже имеет дело не с
исходным полем, а с матричным представлением графа, в котором смежность двух
вершин означает достижимость за один ход одной вершины из другой.\\
Далее, от стартовой ячейки порождается шаг в соседнюю ячейку, при этом
проверяется, не помечена ли она уже.\\
Соседние ячейки в традиционных областях применения алгоритма принято
классифицировать двояко: в смысле окрестности Мура и
окрестности фон Неймана (в окрестности фон Неймана
соседними ячейками считаются только 4 ячейки по вертикали и горизонтали, в
окрестности Мура - все 8 ячеек, включая диагональные). Для нашей задачи ячейки
будут являться ``соседними'', если конь может перейти из одной ячейки в другую
и наоборот.\\
При выполнении условий, в атрибут ячейки записывается число, равное количеству шагов от
стартовой ячейки. Каждая
ячейка, меченая числом шагов от стартовой ячейки, порождает ``новую волну'' в 
соседние ячейки. Очевидно, что при таком переборе
будет найден путь от начальной ячейки к конечной, либо очередной шаг из любой
порождённой в пути ячейки будет невозможен.\\
Восстановление кратчайшего пути происходит в обратном направлении: при выборе
ячейки от финишной ячейки к стартовой на каждом шаге выбирается ячейка,
имеющая атрибут расстояния от стартовой на единицу меньше текущей ячейки.
Очевидно, что таким образом находится кратчайший путь между парой заданных
ячеек.
\pagebreak
