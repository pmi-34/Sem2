\section{Постановка задачи}
\par
На шахматной доске $N\times{N}$ в клетке $(x_1,y_1)$ стоит голодный шахматный конь. Он
хочет попасть в клетку $(x_2,y_2)$, где растет вкусная шахматная трава. Какое
наименьшее количество ходов он должен для этого сделать?\\
{\bf Формат входных данных:}\\
Входной файл содержит пять чисел: $N, x_1, y_1, x_2, y_2 (5 \le N \le 20, 1
\le x_1,y_1,x_2,y_2 \le N)$. Левая верхняя клетка доски имеет координаты (1,1), правая
нижняя - $(N,N)$. \\
{\bf Формат выходных данных:}\\
Первая строка выходного файла должна содержать единственное число $K$ -
наименьшее необходимое число ходов коня. В каждой из следующих $K+1$ строк
должно быть записано 2 числа - координаты очередной клетки в пути коня.
\pagebreak

