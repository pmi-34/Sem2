\section{Определение идеи алгоритма, выбор методов решения и структур данных}
\par
Для начала определимся с типами данных. По условию задачи необходимо
использовать в качестве основы алгоритма граф. Очевидно, что наиболее
эффективной и наименее ресурсоемкой будет реализация волнового алгоритма на матричном
представлении графа, на нем и остановимся. Итоговый же путь будет представлять
собой просто одномерный массив номеров вершин.
\par
Ввод данных по условию задачи необходимо организовать через файл; также стоит предусмотреть
возможность интерактивной корректировки входных данных пользователем - это
удобно и довольно приятно.
\par
По условию, вывод результата необходимо организовать в файл; естественно, что
нужно показать выводимые данные пользователю - для этого предназначена
визуализация (и, собственно, для этого и выбрана среда Delphi).
\par
Программа будет состоять из двух модулей:
\begin{itemize}
\item модуль ввода/вывода и отображения
данных
\item модуль, выполняющий собственно волновой алгоритм
\end{itemize}
Модуль ввода/вывода представляет собой графическое приложение, при старте
считывающее данные из файла $input.txt$ (если он существует), отображающее их
и дающее пользователю возможность в интерактивном режиме их изменять, наблюдая
при этом изменение ответа программы. По завершению работы программы в выходной
файл будут записаны результаты работы алгоритма.
\par
В качестве отдельного Unit-файла был выделен модуль, реализующий волновой
алгоритм. В данном модуле две основные процедуры:
\begin{itemize}
\item инициализация и распространение волны
\item обратный ход волны
\end{itemize},
и несколько вспомогательных - вывод матрицы графа в файл, распечатка
промежуточных значений и прочее.
\par
Пару слов насчет интерфейса. 
По условию задачи, у нас есть шахматное поле размером $NxN$, координаты коня и
координаты травы. Соответственно, разумно будет отображать именно шахматное
поле, с конем и травой на соответствующих позициях. Кликом правой кнопки мыши
по полю можно будет менять позицию коня, левой - позицию травы. Отдельно стоит
предусмотреть ввод размерности шахматной доски, естественно, в допустимых
пределах.
\pagebreak

