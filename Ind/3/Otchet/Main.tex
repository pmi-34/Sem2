% Это комментарий
% Комментарии начинаются с символа %
% Формат А4, стиль - отчет
\documentclass[a4paper,12pt]{report}
% Поля - 2 сантиметра с каждой стороны, без места на подшивку
\usepackage[left=2cm,right=2cm,
    top=2cm,bottom=2cm,bindingoffset=0cm]{geometry}
% Входная кодировка - UTF-8
\usepackage[utf8]{inputenc}
% Язык текста - русский и английский. Русский - основной
\usepackage[english,russian]{babel}
% Для команды \Box и других математических
\usepackage{amsfonts}
% Для вставки изображений
\usepackage{graphicx}
% Для вставки исходного кода
% \usepackage{listings}
% Настройки листингов
% \lstloadlanguages{[x86masm]Assembler}
% \lstset{
% language=[x86masm]Assembler,
% extendedchars=\true,
% inputencoding=utf8,
% commentstyle=\itshape,
% stringstyle=\bf,
% belowcaptionskip=5pt
% }
% Нумерованные списки в виде 1.1
\renewcommand{\labelenumii}{\arabic{enumi}.\arabic{enumii}.}
\renewcommand{\labelenumiii}{\arabic{enumi}.\arabic{enumii}.\arabic{enumiii}.}
\renewcommand{\labelenumiv}{\arabic{enumi}.\arabic{enumii}.\arabic{enumiii}.\arabic{enumiv}.}
% Первый абзац должен начинаться с красной строки
\usepackage{indentfirst}
\usepackage{float}
\restylefloat{figure}
% Мы не используем главы, поэтому подправим номера секций
\renewcommand{\thesection}{\arabic{section}.}
% Исправим досадную ошибку с неразрывным пробелом в UTF-8
\DeclareUnicodeCharacter{00A0}{~}
\begin{document}
% Титульная страница
\begin{titlepage}
\begin{center}

% Верх страницы

\textsc{\large \bf Федеральное агентство по образованию\\
ГОУ ВПО «Пермский государственный национально-исследовательский университет»\\
Кафедра прикладной математики и информатики}\\[1.5cm]

\vfill

% Заголовок
{ \Large \bfseries Индивидуальная работа \No4\\
``Графы и среда Delphi''\\
\em по дисциплине ``Основы программирования'' \\[0.4cm] }

% Автор и рецензент
\begin{flushright}
\begin{minipage}{0.5 \textwidth}
\begin{flushleft} \large
\emph{\bf Выполнил:} \\
Студент 1-го курса \\
механико-математического~ф-та \\
\textsc{Протопопов У.Г.},\\
группа ПМИ-0.\\
\emph{\bf Проверил:} \\
Доцент кафедры ПМИ ПГУ,\\
к.т.н {\underline{\hspace{50 pt}}} \textsc{Перескокова О.И}\\
\today
\end{flushleft}
\end{minipage}
\end{flushright}

\vfill

% Низ страницы
Пермь \the\year

\end{center}
\end{titlepage}


\setcounter{page}{2}
\tableofcontents
\pagebreak
\section{Постановка задачи}
\par
На шахматной доске $N\times{N}$ в клетке $(x_1,y_1)$ стоит голодный шахматный конь. Он
хочет попасть в клетку $(x_2,y_2)$, где растет вкусная шахматная трава. Какое
наименьшее количество ходов он должен для этого сделать?\\
{\bf Формат входных данных:}\\
Входной файл содержит пять чисел: $N, x_1, y_1, x_2, y_2 (5 \le N \le 20, 1
\le x_1,y_1,x_2,y_2 \le N)$. Левая верхняя клетка доски имеет координаты (1,1), правая
нижняя - $(N,N)$. \\
{\bf Формат выходных данных:}\\
Первая строка выходного файла должна содержать единственное число $K$ -
наименьшее необходимое число ходов коня. В каждой из следующих $K+1$ строк
должно быть записано 2 числа - координаты очередной клетки в пути коня.
\pagebreak


\section{Теория}
\DeclareRobustCommand{\c}[1]{
\begin{center}
{#1}
\end{center}
}
\par
{\bf Связный список} — базовая динамическая структура данных, состоящая из 
узлов, каждый из которых содержит как собственно данные, так и одну или две
 ссылки (``связки'') на следующий и/или предыдущий узел списка. 
Принципиальным преимуществом перед массивом является структурная гибкость: 
порядок элементов связного списка может не совпадать с порядком расположения 
элементов данных в памяти компьютера, а порядок обхода списка всегда явно 
задаётся его внутренними связями.

{\bf Линейный однонаправленный список} - это структура данных, состоящая 
из элементов одного типа, связанных между собой единственной связью -
от предыдущего к следующему.
На практике линейные списки обычно реализуются при помощи массивов и 
связных списков. Иногда термин ``список'' неформально используется 
также как синоним понятия ``связный список''.

Характеристики:

\begin{enumerate}
  \item {\bf Длина списка}. Количество элементов в списке;
  \item Списки могут быть {\bf типизированными} или {\bf нетипизированными}. 
        Если 
        список типизирован, то тип его элементов задан, и все его элементы 
        должны иметь типы, совместимые с заданным типом элементов списка;
        Обычно списки являются типизированными;
  \item Список может быть {\bf сортированным} или {\bf несортированным};
  \item В зависимости от реализации может быть возможен {\bf произвольный доступ} к 
        элементам списка.
\end{enumerate}
\pagebreak

\section{Определение идеи алгоритма, выбор методов решения и структур данных}
\par
Для начала определимся с типами данных. По условию задачи необходимо
использовать в качестве основы алгоритма граф. Очевидно, что наиболее
эффективной и наименее ресурсоемкой будет реализация волнового алгоритма на матричном
представлении графа, на нем и остановимся. Итоговый же путь будет представлять
собой просто одномерный массив номеров вершин.
\par
Ввод данных по условию задачи необходимо организовать через файл; также стоит предусмотреть
возможность интерактивной корректировки входных данных пользователем - это
удобно и довольно приятно.
\par
По условию, вывод результата необходимо организовать в файл; естественно, что
нужно показать выводимые данные пользователю - для этого предназначена
визуализация (и, собственно, для этого и выбрана среда Delphi).
\par
Программа будет состоять из двух модулей:
\begin{itemize}
\item модуль ввода/вывода и отображения
данных
\item модуль, выполняющий собственно волновой алгоритм
\end{itemize}
Модуль ввода/вывода представляет собой графическое приложение, при старте
считывающее данные из файла $input.txt$ (если он существует), отображающее их
и дающее пользователю возможность в интерактивном режиме их изменять, наблюдая
при этом изменение ответа программы. По завершению работы программы в выходной
файл будут записаны результаты работы алгоритма.
\par
В качестве отдельного Unit-файла был выделен модуль, реализующий волновой
алгоритм. В данном модуле две основные процедуры:
\begin{itemize}
\item инициализация и распространение волны
\item обратный ход волны
\end{itemize},
и несколько вспомогательных - вывод матрицы графа в файл, распечатка
промежуточных значений и прочее.
\par
Пару слов насчет интерфейса. 
По условию задачи, у нас есть шахматное поле размером $NxN$, координаты коня и
координаты травы. Соответственно, разумно будет отображать именно шахматное
поле, с конем и травой на соответствующих позициях. Кликом правой кнопки мыши
по полю можно будет менять позицию коня, левой - позицию травы. Отдельно стоит
предусмотреть ввод размерности шахматной доски, естественно, в допустимых
пределах.
\pagebreak


\section{Детальное описание алгоритма}
\DeclareRobustCommand{\l}[1]{
\item {[}{\it #1}{]} 
}
Алгоритм состоит из следующих основных процедур:
\begin{itemize}
\item TMainForm.FormCreate - задает начальные параметры программы, считывая
их, если это возможно, из входного файла
\item TMainForm.FormPaint - прорисовывает в окне программы все основные
элементы, как-то поле, коня и траву
\item TMainForm.FieldPaintBoxMouseDown - обработка события нажатия кнопки мыши
на поле
\item BuildHorseMatrix2 - построение матрицы графа, инициализация + прямой ход
алгоритма
\item ReverseWave - обратный ход алгоритма 
\item TMainForm.FormClose - освобождение ресурсов и запись результата в
выходной файл
\end{itemize}

Детальное описание:
\begin{enumerate}
\item При запуске программы вызывается TMainForm.FormCreate. Она подгружает
графические ресурсы с диска и пытается прочитать начальные значения параметров
из входного файла. Если считывание не удалось, или во время него произошла
ошибка, параметрам присваиваются значения по умолчанию.
\item Основной цикл работы программы включает в себя два главных компонента:
отображение результата работы алгоритма и обработку действий пользователя
\begin{enumerate}
\item За отображение результата отвечает процедура TMainForm.FormPaint. Она
прорисовывает в окне прогаммы необходимые компоненты:
\begin{enumerate}
\item Сначала рисуется поле $NxN$ клеток-квадратов, клетки раскрашены в цвет
шахматной доски
\item Затем поверх поля отображается текущая позиция коня и травы, конь и
трава - спрайты, подгружаемые процедурой FormCreate
\item После этого рисуется текущий путь коня, начиная с его позиции и
заканчивая позицией, на которой размещена трава. Путь отображается
последовательностью символов ``галочка'', этот символ также является спрайтом
\item В самом конце производится проверка, нужно ли отображать матрицу
промежуточных результатов; если да - поверх полученного ранее изображения
рисуется матрица (для каждой клетки - расстояние от нее до клетки с конем)
\end{enumerate}
\item За взаимодействие с пользователем отвечает процедура
TMainForm.FieldPaintBoxMouseDown:
\begin{enumerate}
\item Если произошел щелчок левой кнопкой мыши, вычисляем новые координаты
коня, и, если они допустимы (не выходят за пределы поля), переносим его на
новую позицию
\item Если произошел щелчок правой кнопкой, то вычисляем новые координаты
травы и, соответственно, при их корректности переносим траву
\item По окончании переноса коня и травы заново запускаем алгоритм и форсируем
перерисовку
\end{enumerate}
\item Также на форме имеется элемент для ввода размерности поля. При его
изменении меняется размерность поля, а также корректируются координаты коня и
травы, если они выходят за границы поля новой размерности. По окончании
обработки события форсируется обновление данных и перерисовка поля.
\item По завершении работы программы результирующие данные (количество ходов
коня и координаты клеток, по которым пройдет его маршрут) записываются в
результирующий файл.
\end{enumerate}
\end{enumerate}

\pagebreak


\section{Тестирование программы}
\newcounter{testcnt}
\DeclareRobustCommand{\test}[3]{
  \addtocounter{testcnt}{1}
  \par
  Тест \arabic{testcnt}: {#1}\\
  {\it Входные данные:} {#2}\\
  {\it Выходные данные:} {#3}\\
  {\it Ожидаемый результат:} {#3}\\
  \begin {figure}[H] 
  \centering 
  \includegraphics[scale=0.5]{test\arabic{testcnt}.png} 
  \caption{Скриншот} 
  \label{fig:scr\arabic{testcnt}} 
  \end {figure}
}
\par
Тесты для программы были разбиты на несколько групп таким образом, чтобы
покрыть одновременно максимальное количество типов входных данных и вариантов
выполнения тела программы.\\
%\test{}{}{}{}
\test{Результат выполнения программы с нормальными входными данными}{Смотри
рисунок}{Корректные выходные данные}
\test{Выполнение программы с корректными входными данными}{Смотри предыдущий
рисунок}{Правильное отображение элементов}
\test{Обработка ситуации отсутствия входного файла}{Входной файл
отсутствует}{Всем параметрам присваивается значение по умолчанию}
\test{Некорректные входные данные}{Файл с некорректными данными}{Всем
параметрам присваиваются значения по умолчанию}
\test{Перемещение элементов}{Пользователь кликом мыши перемещает коня и
траву}{Конь и трава корректно перемещаются, путь пересчитывается}
\test{Динамическое изменение размера поля на больший}{Пользователь задает новый размер
поля}{Поле меняет размер, конь и трава не перемещаются; возможно,
пересчитывается путь}
\test{Динамическое изменение размера поля на меньший}{Пользователь задает
новый размер поля, конь и/или трава в него не помещаются}{Конь и/или трава
меняют свое положение на допустимые}
\pagebreak

\section{Заключение}
\par
В итоге данной индивидуальной работы нам удалось написать программу, которая
решает поставленную перед ней задачу. Кроме того, мы познакомились с
принципами разработки и тестирования ПО, научились применять их во благо.
\pagebreak

\section{Текст программы}
Ниже приведен полный исходный текст программы, созданной в текстовом 
редакторе Vim 7.3. Программа успешно собирается с использованием FPC 2.6.2
для 64-разрядного Unix.\\
\fbox{Основная программа:}\\
% Generator: GNU source-highlight, by Lorenzo Bettini, http://www.gnu.org/software/src-highlite
\noindent
\mbox{}\textbf{program}\ \textbf{Ind2}\ (input,\ output); \\
\mbox{} \\
\mbox{}\textit{\{\$DEFINE\ DEBUG\}} \\
\mbox{} \\
\mbox{}\textbf{uses} \\
\mbox{}\ \ StackUtil,\ SysUtils; \\
\mbox{} \\
\mbox{}\textit{\{Определяет,\ является\ ли\ аргумент\ числом\}} \\
\mbox{}\textbf{function}\ \textbf{isdigit}(X\ :\ char)\ :\ boolean; \\
\mbox{}\textbf{begin} \\
\mbox{}\ \ isdigit\ :=\ (X\ \textgreater{}=\ \texttt{'0'})\ \textbf{AND}\ (X\ \textless{}=\ \texttt{'9'}); \\
\mbox{}\textbf{end}; \\
\mbox{} \\
\mbox{}\textit{\{Определяет,\ является\ ли\ аргумент\ символом\}} \\
\mbox{}\textbf{function}\ \textbf{isalpha}(X\ :\ char)\ :\ boolean; \\
\mbox{}\textbf{begin} \\
\mbox{}\ \ isalpha\ :=\ (\textbf{upcase}(X)\ \textgreater{}=\ \texttt{'A'})\ \textbf{AND}\ (\textbf{upcase}(X)\ \textless{}=\ \texttt{'Z'}); \\
\mbox{}\textbf{end}; \\
\mbox{} \\
\mbox{}\textit{\{Переводит\ часть\ строки\ в\ число\}} \\
\mbox{}\textbf{function}\ \textbf{GetN}(S\ :\ string;\ \textbf{var}\ I\ :\ integer)\ :\ integer; \\
\mbox{}\textbf{var} \\
\mbox{}\ \ R\ :\ integer; \\
\mbox{}\textbf{begin} \\
\mbox{}\ \ R\ :=\ 0; \\
\mbox{} \\
\mbox{}\ \ \textbf{while}\ ((I\ \textless{}=\ \textbf{length}(S))\ \textbf{AND}\ \textbf{isdigit}(S[i]))\ \textbf{do}\ \textbf{begin} \\
\mbox{}\ \ \ \ R\ :=\ R*10\ +\ \textbf{ord}(S[I])\ -\ \$30;\ \textit{//\ ord('0');} \\
\mbox{}\ \ \ \ \textbf{inc}(I); \\
\mbox{}\ \ \textbf{end}; \\
\mbox{} \\
\mbox{}\ \ GetN\ :=\ R; \\
\mbox{}\textbf{end}; \\
\mbox{} \\
\mbox{}\textit{\{Создание\ дерева\ из\ инфиксной\ записи\}} \\
\mbox{}\textbf{procedure}\ \textbf{CreateInfixTree}(\textbf{var}\ Root\ :\ PNode;\ S\ :\ string); \\
\mbox{}\textbf{var} \\
\mbox{}\ \ Queue,\ \textit{\{голова\ очереди\}} \\
\mbox{}\ \ QEnd,\ \textit{\{хвост\ очереди\}} \\
\mbox{}\ \ Last\ \textit{\{последний\ распознанный\ элемент\}} \\
\mbox{}\ \ :\ PNode; \\
\mbox{} \\
\mbox{}\ \ \textit{\{Добавление\ элемента\ в\ очередь\}} \\
\mbox{}\ \ \textbf{procedure}\ \textbf{AddToQueue}(P\ :\ PNode); \\
\mbox{}\ \ \textbf{var} \\
\mbox{}\ \ \ \ Q\ :\ PNode; \\
\mbox{}\ \ \textbf{begin} \\
\mbox{}\ \ \ \ \textbf{if}\ (Queue\ =\ nil)\ \textbf{then}\ \textbf{begin} \\
\mbox{}\ \ \ \ \ \ Queue\ :=\ P; \\
\mbox{}\ \ \ \ \ \ QEnd\ :=\ P; \\
\mbox{}\ \ \ \ \textbf{end}\ \textbf{else}\ \textbf{begin} \\
\mbox{}\ \ \ \ \ \ QEnd\textasciicircum{}.Next\ :=\ P; \\
\mbox{}\ \ \ \ \ \ QEnd\ :=\ P; \\
\mbox{}\ \ \ \ \textbf{end}; \\
\mbox{}\ \ \textbf{end}; \\
\mbox{} \\
\mbox{}\ \ \textit{\{Создание\ узла\ дерева\}} \\
\mbox{}\ \ \textbf{function}\ \textbf{Build}(Data\ :\ integer;\ T\ :\ RType)\ :\ PNode; \\
\mbox{}\ \ \textbf{var} \\
\mbox{}\ \ \ \ Q\ :\ PNode; \\
\mbox{}\ \ \textbf{begin} \\
\mbox{}\ \ \ \ \textbf{New}(Q); \\
\mbox{}\ \ \ \ Q\textasciicircum{}.Next\ :=\ nil; \\
\mbox{}\ \ \ \ Q\textasciicircum{}.Left\ :=\ nil; \\
\mbox{}\ \ \ \ Q\textasciicircum{}.Right\ :=\ nil; \\
\mbox{}\ \ \ \ Q\textasciicircum{}.Data\ :=\ Data; \\
\mbox{}\ \ \ \ Q\textasciicircum{}.T\ :=\ T; \\
\mbox{}\ \ \ \ Last\ :=\ Q; \\
\mbox{}\ \ \ \ Build\ :=\ Q; \\
\mbox{}\ \ \textbf{end}; \\
\mbox{} \\
\mbox{}\textbf{var} \\
\mbox{}\ \ i\ :\ integer; \\
\mbox{}\ \ Q\ :\ PNode; \\
\mbox{}\textbf{begin} \\
\mbox{}\ \ i\ :=\ 1; \\
\mbox{}\ \ Queue\ :=\ nil; \\
\mbox{}\ \ QEnd\ :=\ nil; \\
\mbox{}\ \ Last\ :=\ nil; \\
\mbox{} \\
\mbox{}\ \ ResetStack; \\
\mbox{} \\
\mbox{}\ \ \textit{\{Преобразование\ выразения\ в\ обратную\ польскую\ запись\}} \\
\mbox{}\ \ \textbf{while}\ ((i\ \textless{}=\ \textbf{Length}(S))\ \textbf{and}\ \textbf{not}\ IsError)\ \textbf{do}\ \textbf{begin} \\
\mbox{}\ \ \ \ \textit{\{Пропустим\ пробелы\}} \\
\mbox{}\ \ \ \ \textbf{if}\ (S[i]\ \textless{}\textgreater{}\ \texttt{'\ '})\ \textbf{then}\ \textbf{begin} \\
\mbox{}\ \ \ \ \ \ \textit{\{После\ символа\ переменной\ должен\ быть\ только\ знак\ или\ конец\ строки\}} \\
\mbox{}\ \ \ \ \ \ \textbf{if}\ (Last\ \textless{}\textgreater{}\ nil)\ \textbf{and}\ (Last\textasciicircum{}.T\ =\ Variable)\ \textbf{and}\ (\textbf{isdigit}(S[i])\ \textbf{or} \\
\mbox{}\ \ \ \ \ \ \textbf{isalpha}(S[i])\ \textbf{or}\ (S[i]\ =\ \texttt{'('}))\ \textbf{then} \\
\mbox{}\ \ \ \ \ \ \ \ \textbf{Error}(\texttt{'Неправильное\ задание\ переменной'},\ S,\ i) \\
\mbox{}\ \ \ \ \ \ \textit{\{Добавим\ число\ в\ очередь\}} \\
\mbox{}\ \ \ \ \ \ \textbf{else}\ \textbf{if}\ \textbf{isdigit}(S[i])\ \textbf{then}\ \textbf{begin} \\
\mbox{}\ \ \ \ \ \ \ \ \textbf{AddToQueue}(\textbf{Build}(\textbf{GetN}(S,\ I),\ Constant)); \\
\mbox{}\ \ \ \ \ \ \ \ \textbf{dec}(i); \\
\mbox{}\ \ \ \ \ \ \textbf{end} \\
\mbox{}\ \ \ \ \ \ \textit{\{Добавим\ переменную\ в\ очередь\}} \\
\mbox{}\ \ \ \ \ \ \textbf{else}\ \textbf{if}\ \textbf{isalpha}(S[i])\ \textbf{then} \\
\mbox{}\ \ \ \ \ \ \ \ \textbf{AddToQueue}(\textbf{Build}(\textbf{ord}(S[i]),\ Variable)) \\
\mbox{}\ \ \ \ \ \ \textit{\{Кладем\ в\ стек\ скобку\}} \\
\mbox{}\ \ \ \ \ \ \textbf{else}\ \textbf{if}\ (S[i]\ =\ \texttt{'('})\ \textbf{then} \\
\mbox{}\ \ \ \ \ \ \ \ \textbf{Push}(\textbf{Build}(0,\ Bracket)) \\
\mbox{}\ \ \ \ \ \ \textit{\{Извлекаем\ из\ стека\ все\ до\ открывающей\ скобки\}} \\
\mbox{}\ \ \ \ \ \ \textbf{else}\ \textbf{if}\ (S[i]\ =\ \texttt{')'})\ \textbf{then}\ \textbf{begin} \\
\mbox{}\ \ \ \ \ \ \ \ Q\ :=\ Pop; \\
\mbox{}\ \ \ \ \ \ \ \ \textbf{while}\ (Q\ \textless{}\textgreater{}\ nil)\ \textbf{AND}\ (Q\textasciicircum{}.T\ \textless{}\textgreater{}\ Bracket)\ \textbf{do}\ \textbf{begin} \\
\mbox{}\ \ \ \ \ \ \ \ \ \ \textbf{AddToQueue}(Q); \\
\mbox{}\ \ \ \ \ \ \ \ \ \ Q\ :=\ Pop; \\
\mbox{}\ \ \ \ \ \ \ \ \textbf{end}; \\
\mbox{}\ \ \ \ \ \ \ \ \textbf{if}\ (Q\ =\ nil)\ \textbf{then} \\
\mbox{}\ \ \ \ \ \ \ \ \ \ \textbf{Error}(\texttt{'В\ выражении\ несбалансированы\ скобки'},\ S,\ I) \\
\mbox{}\ \ \ \ \ \ \ \ \textbf{else} \\
\mbox{}\ \ \ \ \ \ \ \ \ \ \textbf{Dispose}(Q);\ \textit{//\ Q\textasciicircum{}.T\ =\ Bracket} \\
\mbox{}\ \ \ \ \ \ \textbf{end} \\
\mbox{}\ \ \ \ \ \ \textit{\{Извлекаем\ все\ операции\ того\ же\ приоритета\}} \\
\mbox{}\ \ \ \ \ \ \textbf{else}\ \textbf{if}\ ((S[i]\ =\ \texttt{'*'})\ \textbf{or}\ (S[i]\ =\ \texttt{'/'}))\ \textbf{then}\ \textbf{begin} \\
\mbox{}\ \ \ \ \ \ \ \ \textbf{while}\ ((\textbf{not}\ IsClear)\ \textbf{AND}\  \\
\mbox{}\ \ \ \ \ \ \ \ \ \ \ \ \ \ \ ((ReadTop\textasciicircum{}.Data\ =\ \textbf{ord}(\texttt{'*'}))\ \textbf{OR}\ (ReadTop\textasciicircum{}.Data\ =\ \textbf{ord}(\texttt{'/'}))))\ \textbf{do} \\
\mbox{}\ \ \ \ \ \ \ \ \ \ \textbf{AddToQueue}(Pop); \\
\mbox{}\ \ \ \ \ \ \ \ \textbf{Push}(\textbf{Build}(\textbf{ord}(S[i]),\ Operation)); \\
\mbox{}\ \ \ \ \ \ \textbf{end} \\
\mbox{}\ \ \ \ \ \ \textit{\{Извлекаем\ все\ операции\ того\ же\ или\ более\ высокого\ приоритета\}} \\
\mbox{}\ \ \ \ \ \ \textbf{else}\ \textbf{if}\ ((S[i]\ =\ \texttt{'+'})\ \textbf{or}\ (S[i]\ =\ \texttt{'-'}))\ \textbf{then}\ \textbf{begin} \\
\mbox{}\ \ \ \ \ \ \ \ \textbf{if}\ ((S[i]\ =\ \texttt{'-'})\ \textbf{and}\  \\
\mbox{}\ \ \ \ \ \ \ \ \ \ \ \ ((Last\ =\ nil)\ \textbf{or}\ (Last\textasciicircum{}.T\ =\ Bracket)\ \textbf{or}\ (Last\textasciicircum{}.T\ =\ Operation)))\ \textbf{then}\ \textbf{begin} \\
\mbox{}\ \ \ \ \ \ \ \ \ \ \ \textit{\{Минус\ унарный,\ ничего\ не\ извлекаем\ -\ просто\ добавим\ 0\ перед\ ним\}} \\
\mbox{}\ \ \ \ \ \ \ \ \ \ \ \textbf{AddToQueue}(\textbf{Build}(0,\ Constant)); \\
\mbox{}\ \ \ \ \ \ \ \ \textbf{end}\ \textbf{else}\ \textbf{while}\ ((\textbf{not}\ IsClear)\ \textbf{AND} \\
\mbox{}\ \ \ \ \ \ \ \ \ \ \ ((ReadTop\textasciicircum{}.Data\ =\ \textbf{ord}(\texttt{'*'}))\ \textbf{or} \\
\mbox{}\ \ \ \ \ \ \ \ \ \ \ \ (ReadTop\textasciicircum{}.Data\ =\ \textbf{ord}(\texttt{'+'}))\ \textbf{or} \\
\mbox{}\ \ \ \ \ \ \ \ \ \ \ \ (ReadTop\textasciicircum{}.Data\ =\ \textbf{ord}(\texttt{'-'}))\ \textbf{or} \\
\mbox{}\ \ \ \ \ \ \ \ \ \ \ \ (ReadTop\textasciicircum{}.Data\ =\ \textbf{ord}(\texttt{'/'})) \\
\mbox{}\ \ \ \ \ \ \ \ \ \ \ ))\ \textbf{do} \\
\mbox{}\ \ \ \ \ \ \ \ \ \ \textbf{AddToQueue}(Pop);\ \ \ \  \\
\mbox{}\ \ \ \ \ \ \ \ \textbf{Push}(\textbf{Build}(\textbf{ord}(S[i]),\ Operation)); \\
\mbox{}\ \ \ \ \ \ \textbf{end}\ \textbf{else}\ \textit{\{Нераспознанный\ символ\}} \\
\mbox{}\ \ \ \ \ \ \ \ \textbf{Error}(\texttt{'Неизвестный\ символ'},\ S,\ I); \\
\mbox{}\ \ \ \ \textbf{end}; \\
\mbox{}\ \ \ \ \textbf{inc}(i); \\
\mbox{}\ \ \textbf{end}; \\
\mbox{} \\
\mbox{}\ \ I\ :=\ 0; \\
\mbox{}\ \ \textit{\{Извлечем\ из\ стека\ оставшиеся\ элементы\}} \\
\mbox{}\ \ \textbf{while}\ (\textbf{not}\ IsClear)\ \textbf{do}\ \textbf{begin} \\
\mbox{}\ \ \ \ Q\ :=\ Pop; \\
\mbox{}\ \ \ \ \textbf{if}\ (Q\ \textless{}\textgreater{}\ nil)\ \textbf{and}\ (Q\textasciicircum{}.T\ =\ Bracket)\ \textbf{then} \\
\mbox{}\ \ \ \ \ \ \textbf{Error}(\texttt{'В\ выражении\ не\ закрыты\ скобки'},\ \texttt{''},\ I) \\
\mbox{}\ \ \ \ \textbf{else} \\
\mbox{}\ \ \ \ \ \ \textbf{AddToQueue}(Q); \\
\mbox{}\ \ \textbf{end}; \\
\mbox{} \\
\mbox{}\ \ \textit{\{\$IFDEF\ DEBUG\}} \\
\mbox{}\ \ Q\ :=\ Queue; \\
\mbox{}\ \ \textbf{while}\ (Q\ \textless{}\textgreater{}\ nil)\ \textbf{do}\ \textbf{begin} \\
\mbox{}\ \ \ \ \textbf{if}\ (Q\textasciicircum{}.T\ =\ Operation)\ \textbf{OR}\ (Q\textasciicircum{}.T\ =\ Variable)\ \textbf{then} \\
\mbox{}\ \ \ \ \ \ \textbf{write}\ (\textbf{chr}(Q\textasciicircum{}.Data),\ \texttt{'\ '}) \\
\mbox{}\ \ \ \ \textbf{else} \\
\mbox{}\ \ \ \ \ \ \textbf{write}(Q\textasciicircum{}.Data,\ \texttt{'\ '}); \\
\mbox{} \\
\mbox{}\ \ \ \ Q\ :=\ Q\textasciicircum{}.Next; \\
\mbox{}\ \ \textbf{end}; \\
\mbox{}\ \ \textbf{writeln}; \\
\mbox{}\ \ \textit{\{\$ENDIF\}} \\
\mbox{}\ \  \\
\mbox{}\ \ \textit{\{Если\ польская\ запись\ успешно\ составлена,\ строим\ дерево\}} \\
\mbox{}\ \ \textbf{if}\ (\textbf{not}\ IsError)\ \textbf{then}\ \textbf{begin} \\
\mbox{}\ \ \ \ \textit{\{Строим\ дерево\}} \\
\mbox{}\ \ \ \ Q\ :=\ Queue; \\
\mbox{}\ \ \ \ \textbf{while}\ (Q\ \textless{}\textgreater{}\ nil)\ \textbf{do}\ \textbf{begin} \\
\mbox{}\ \ \ \ \ \ \textbf{if}\ (Q\textasciicircum{}.T\ =\ Operation)\ \textbf{then}\ \textbf{begin} \\
\mbox{}\ \ \ \ \ \ \ \ Q\textasciicircum{}.Right\ :=\ Pop; \\
\mbox{}\ \ \ \ \ \ \ \ Q\textasciicircum{}.Left\ :=\ Pop; \\
\mbox{}\ \ \ \ \ \ \textbf{end}; \\
\mbox{} \\
\mbox{}\ \ \ \ \ \ \textbf{Push}(Q); \\
\mbox{}\ \ \ \  \\
\mbox{}\ \ \ \ \ \ Q\ :=\ Q\textasciicircum{}.Next; \\
\mbox{}\ \ \ \ \textbf{end}; \\
\mbox{} \\
\mbox{}\ \ \ \ Root\ :=\ Pop; \\
\mbox{}\ \ \textbf{end}\ \textbf{else} \\
\mbox{}\ \ \ \ Root\ :=\ nil; \\
\mbox{}\textbf{end}; \\
\mbox{} \\
\mbox{}\textit{\{Вывод\ дерева\ на\ боку,\ зеркально-симметричный\ обратный\ обход,\ RNL\}} \\
\mbox{}\textbf{procedure}\ \textbf{DumpTree}(P\ :\ PNode;\ L\ :\ integer); \\
\mbox{}\textbf{begin} \\
\mbox{}\ \ \textbf{if}\ (P\ \textless{}\textgreater{}\ nil)\ \textbf{then}\ \textbf{begin} \\
\mbox{}\ \ \ \ \textbf{DumpTree}(P\textasciicircum{}.Right,\ L+1); \\
\mbox{}\ \ \ \ \textbf{if}\ (P\textasciicircum{}.T\ =\ Operation)\ \textbf{OR}\ (P\textasciicircum{}.T\ =\ Variable)\ \textbf{then} \\
\mbox{}\ \ \ \ \ \ \textbf{writeln}(\texttt{'\ '}:L*4,\ \textbf{chr}(P\textasciicircum{}.Data)\ :\ 2) \\
\mbox{}\ \ \ \ \textbf{else} \\
\mbox{}\ \ \ \ \ \ \textbf{writeln}(\texttt{'\ '}:L*4,\ P\textasciicircum{}.Data\ :\ 2); \\
\mbox{}\ \ \ \ \textbf{DumpTree}(P\textasciicircum{}.Left,\ L+1); \\
\mbox{}\ \ \textbf{end}; \\
\mbox{}\textbf{end}; \\
\mbox{} \\
\mbox{}\textit{\{Освобождение\ памяти,\ концевой\ обход,\ LRN\}} \\
\mbox{}\textbf{procedure}\ \textbf{FreeTree}(\textbf{var}\ Root\ :\ PNode); \\
\mbox{}\textbf{begin} \\
\mbox{}\ \ \textbf{if}\ (Root\ \textless{}\textgreater{}\ nil)\ \textbf{then}\ \textbf{begin} \\
\mbox{}\ \ \ \ \textbf{FreeTree}(Root\textasciicircum{}.Left); \\
\mbox{}\ \ \ \ \textbf{FreeTree}(Root\textasciicircum{}.Right); \\
\mbox{}\ \ \ \ \textbf{Dispose}(Root); \\
\mbox{}\ \ \ \ Root\ :=\ nil; \\
\mbox{}\ \ \textbf{end}; \\
\mbox{}\textbf{end}; \\
\mbox{} \\
\mbox{}\textit{\{Копирование\ дерева,\ прямой\ обход,\ NLR\}} \\
\mbox{}\textbf{procedure}\ \textbf{CopyTree}(Root\ :\ PNode;\ \textbf{var}\ NewRoot\ :\ PNode); \\
\mbox{}\textbf{begin} \\
\mbox{}\ \ \textbf{if}\ (Root\ \textless{}\textgreater{}\ nil)\ \textbf{then}\ \textbf{begin} \\
\mbox{}\ \ \ \ \textbf{New}(NewRoot); \\
\mbox{}\ \ \ \ NewRoot\textasciicircum{}.Data\ :=\ Root\textasciicircum{}.Data; \\
\mbox{}\ \ \ \ NewRoot\textasciicircum{}.T\ :=\ Root\textasciicircum{}.T; \\
\mbox{}\ \ \ \ \textbf{CopyTree}(Root\textasciicircum{}.Left,\ NewRoot\textasciicircum{}.Left); \\
\mbox{}\ \ \ \ \textbf{CopyTree}(Root\textasciicircum{}.Right,\ NewRoot\textasciicircum{}.Right); \\
\mbox{}\ \ \textbf{end}; \\
\mbox{}\textbf{end}; \\
\mbox{} \\
\mbox{}\textit{\{Ради\ чего\ все\ и\ затевалось\ -\ упрощение\ выражения\}} \\
\mbox{}\textbf{procedure}\ \textbf{Simplify}(\textbf{var}\ Root\ :\ PNode); \\
\mbox{}\textbf{var} \\
\mbox{}\ \ Q\ :\ PNode; \\
\mbox{}\textbf{begin} \\
\mbox{}\ \ \textbf{if}\ (Root\ \textless{}\textgreater{}\ nil)\ \textbf{then}\ \textbf{begin} \\
\mbox{}\ \ \ \ \textbf{Simplify}(Root\textasciicircum{}.Left); \\
\mbox{}\ \ \ \ \textbf{Simplify}(Root\textasciicircum{}.Right); \\
\mbox{} \\
\mbox{}\ \ \ \ \textbf{if}\ ((Root\textasciicircum{}.T\ =\ Operation)\ \textbf{and}\ (Root\textasciicircum{}.Data\ =\ \textbf{ord}(\texttt{'*'})))\ \textbf{then} \\
\mbox{}\ \ \ \ \ \ \textbf{if}\ ((Root\textasciicircum{}.Left\ \textless{}\textgreater{}\ nil)\ \textbf{and}\ (Root\textasciicircum{}.Left\textasciicircum{}.T\ =\ Operation)\ \textbf{and} \\
\mbox{}\ \ \ \ \ \ \ \ \ \ ((Root\textasciicircum{}.Left\textasciicircum{}.Data\ =\ \textbf{ord}(\texttt{'+'}))\ \textbf{or}\ (Root\textasciicircum{}.Left\textasciicircum{}.Data\ =\ \textbf{ord}(\texttt{'-'})))) \\
\mbox{}\ \ \ \ \ \ \ \ \ \ \textbf{then}\ \textbf{begin} \\
\mbox{}\ \ \ \ \ \ \ \ \ \ \textit{\{Мы\ в\ узле-операции\ умножения,\ левый\ потомок\ -\ сложение\}} \\
\mbox{}\ \ \ \ \ \ \ \ \textbf{New}(Q); \\
\mbox{}\ \ \ \ \ \ \ \ Q\textasciicircum{}.T\ :=\ Operation; \\
\mbox{}\ \ \ \ \ \ \ \ Q\textasciicircum{}.Data\ :=\ \textbf{ord}(\texttt{'*'}); \\
\mbox{}\ \ \ \ \ \ \ \ \textbf{CopyTree}(Root\textasciicircum{}.Right,\ Q\textasciicircum{}.Right); \\
\mbox{} \\
\mbox{}\ \ \ \ \ \ \ \ Q\textasciicircum{}.Left\ :=\ Root\textasciicircum{}.Left\textasciicircum{}.Right; \\
\mbox{}\ \ \ \ \ \ \ \ Root\textasciicircum{}.Left\textasciicircum{}.Right\ :=\ Root\textasciicircum{}.Right; \\
\mbox{} \\
\mbox{}\ \ \ \ \ \ \ \ Root\textasciicircum{}.Data\ :=\ Root\textasciicircum{}.Left\textasciicircum{}.Data; \\
\mbox{}\ \ \ \ \ \ \ \ Root\textasciicircum{}.Left\textasciicircum{}.Data\ :=\ \textbf{ord}(\texttt{'*'}); \\
\mbox{} \\
\mbox{}\ \ \ \ \ \ \ \ Root\textasciicircum{}.Right\ :=\ Q; \\
\mbox{}\ \ \ \ \ \ \textbf{end}\ \textbf{else}\ \textbf{if}\ ((Root\textasciicircum{}.Right\ \textless{}\textgreater{}\ nil)\ \textbf{and}\ (Root\textasciicircum{}.Right\textasciicircum{}.T\ =\ Operation)\ \textbf{and} \\
\mbox{}\ \ \ \ \ \ \ \ \ \ ((Root\textasciicircum{}.Right\textasciicircum{}.Data\ =\ \textbf{ord}(\texttt{'+'}))\ \textbf{or}\ (Root\textasciicircum{}.Right\textasciicircum{}.Data\ =\ \textbf{ord}(\texttt{'-'})))) \\
\mbox{}\ \ \ \ \ \ \ \ \ \ \textbf{then}\ \textbf{begin} \\
\mbox{}\ \ \ \ \ \ \ \ \ \ \textit{\{Правый\ потомок\ -\ сложение\}} \\
\mbox{}\ \ \ \ \ \ \ \ \textbf{New}(Q); \\
\mbox{}\ \ \ \ \ \ \ \ Q\textasciicircum{}.T\ :=\ Operation; \\
\mbox{}\ \ \ \ \ \ \ \ Q\textasciicircum{}.Data\ :=\ \textbf{ord}(\texttt{'*'}); \\
\mbox{}\ \ \ \ \ \ \ \ \textbf{CopyTree}(Root\textasciicircum{}.Left,\ Q\textasciicircum{}.Left); \\
\mbox{} \\
\mbox{}\ \ \ \ \ \ \ \ Q\textasciicircum{}.Right\ :=\ Root\textasciicircum{}.Right\textasciicircum{}.Left; \\
\mbox{}\ \ \ \ \ \ \ \ Root\textasciicircum{}.Right\textasciicircum{}.Left\ :=\ Root\textasciicircum{}.Left; \\
\mbox{} \\
\mbox{}\ \ \ \ \ \ \ \ Root\textasciicircum{}.Data\ :=\ Root\textasciicircum{}.Right\textasciicircum{}.Data; \\
\mbox{}\ \ \ \ \ \ \ \ Root\textasciicircum{}.Right\textasciicircum{}.Data\ :=\ \textbf{ord}(\texttt{'*'}); \\
\mbox{} \\
\mbox{}\ \ \ \ \ \ \ \ Root\textasciicircum{}.Left\ :=\ Q; \\
\mbox{}\ \ \ \ \ \ \textbf{end}; \\
\mbox{}\ \ \textbf{end}; \\
\mbox{}\textbf{end}; \\
\mbox{} \\
\mbox{}\textbf{var} \\
\mbox{}\ \ \textit{\{Значения\ переменных\}} \\
\mbox{}\ \ Var$\_$vals\ :\ \textbf{array}\ [\texttt{'a'}..\texttt{'z'}]\ \textbf{of}\ integer; \\
\mbox{}\ \ \textit{\{Определена\ ли\ переменная\}} \\
\mbox{}\ \ Var$\_$def\ :\ \textbf{array}\ [\texttt{'a'}..\texttt{'z'}]\ \textbf{of}\ boolean; \\
\mbox{} \\
\mbox{}\textbf{function}\ \textbf{DoCalculation}(\textbf{var}\ Root\ :\ PNode)\ :\ real; \\
\mbox{}\textbf{var} \\
\mbox{}\ \ c\ :\ char; \\
\mbox{}\ \ r\ :\ real; \\
\mbox{}\ \ i\ :\ integer; \\
\mbox{}\textbf{begin} \\
\mbox{}\ \ \textbf{if}\ (Root\ \textless{}\textgreater{}\ nil)\ \textbf{and}\ (\textbf{not}\ IsError)\ \textbf{then}\ \textbf{begin} \\
\mbox{}\ \ \ \ c\ :=\ \textbf{chr}(Root\textasciicircum{}.Data); \\
\mbox{}\ \ \ \ \textbf{if}\ (Root\textasciicircum{}.T\ =\ Constant)\ \textbf{then} \\
\mbox{}\ \ \ \ \ \ DoCalculation\ :=\ Root\textasciicircum{}.Data \\
\mbox{}\ \ \ \ \textbf{else}\ \textbf{if}\ (Root\textasciicircum{}.T\ =\ Variable)\ \textbf{then}\ \textbf{begin} \\
\mbox{}\ \ \ \ \ \ \textbf{if}\ (\textbf{not}\ Var$\_$def[c])\ \textbf{then}\ \textbf{begin} \\
\mbox{}\ \ \ \ \ \ \ \ \textbf{write}(\texttt{'Введите\ значение\ переменной\ '},\ c,\ \texttt{':'}); \\
\mbox{}\ \ \ \ \ \ \ \ \textbf{readln}(Var$\_$vals[c]); \\
\mbox{}\ \ \ \ \ \ \ \ Var$\_$def[c]\ :=\ \textbf{true}; \\
\mbox{}\ \ \ \ \ \ \textbf{end}; \\
\mbox{}\ \ \ \ \ \ DoCalculation\ :=\ Var$\_$vals[c]; \\
\mbox{}\ \ \ \ \textbf{end}\ \textbf{else}\ \textbf{begin} \\
\mbox{}\ \ \ \ \ \ \textbf{case}\ c\ \textbf{of} \\
\mbox{}\ \ \ \ \ \ \ \ \texttt{'+'}:\ DoCalculation\ :=\ \textbf{DoCalculation}(Root\textasciicircum{}.Left)\ + \\
\mbox{}\ \ \ \ \ \ \ \ \ \ \ \ \ \ \ \ \ \ \ \ \ \ \ \ \ \ \ \ \ \ \textbf{DoCalculation}(Root\textasciicircum{}.Right); \\
\mbox{}\ \ \ \ \ \ \ \ \texttt{'-'}:\ DoCalculation\ :=\ \textbf{DoCalculation}(Root\textasciicircum{}.Left)\ - \\
\mbox{}\ \ \ \ \ \ \ \ \ \ \ \ \ \ \ \ \ \ \ \ \ \ \ \ \ \ \ \ \ \ \textbf{DoCalculation}(Root\textasciicircum{}.Right); \\
\mbox{}\ \ \ \ \ \ \ \ \texttt{'*'}:\ DoCalculation\ :=\ \textbf{DoCalculation}(Root\textasciicircum{}.Left)\ * \\
\mbox{}\ \ \ \ \ \ \ \ \ \ \ \ \ \ \ \ \ \ \ \ \ \ \ \ \ \ \ \ \ \ \textbf{DoCalculation}(Root\textasciicircum{}.Right); \\
\mbox{}\ \ \ \ \ \ \ \ \texttt{'/'}:\ \textbf{begin} \\
\mbox{}\ \ \ \ \ \ \ \ \ \ \ \ \ \ \ r\ :=\ \textbf{DoCalculation}(Root\textasciicircum{}.Right); \\
\mbox{}\ \ \ \ \ \ \ \ \ \ \ \ \ \ \ i\ :=\ 0; \\
\mbox{}\ \ \ \ \ \ \ \ \ \ \ \ \ \ \ \textbf{if}\ (r\ =\ 0)\ \textbf{then} \\
\mbox{}\ \ \ \ \ \ \ \ \ \ \ \ \ \ \ \ \ \textbf{Error}(\texttt{'Попытка\ деления\ на\ ноль!'},\ \texttt{''},\ i) \\
\mbox{}\ \ \ \ \ \ \ \ \ \ \ \ \ \ \ \textbf{else} \\
\mbox{}\ \ \ \ \ \ \ \ \ \ \ \ \ \ \ \ \ DoCalculation\ :=\ \textbf{DoCalculation}(Root\textasciicircum{}.Left)\ /\ r; \\
\mbox{}\ \ \ \ \ \ \ \ \ \ \ \ \ \textbf{end}; \\
\mbox{}\ \ \ \ \ \ \textbf{end}; \\
\mbox{}\ \ \ \ \textbf{end}; \\
\mbox{}\ \ \textbf{end}\ \textbf{else} \\
\mbox{}\ \ \ \ DoCalculation\ :=\ 0; \\
\mbox{}\textbf{end}; \\
\mbox{} \\
\mbox{}\textbf{function}\ \textbf{Calculate}\ (\textbf{var}\ Root\ :\ PNode)\ :\ real; \\
\mbox{}\textbf{var} \\
\mbox{}\ \ c\ :\ char; \\
\mbox{}\textbf{begin} \\
\mbox{}\ \ \textbf{for}\ c\ :=\ \texttt{'a'}\ \textbf{to}\ \texttt{'z'}\ \textbf{do}\ \textbf{begin} \\
\mbox{}\ \ \ \ Var$\_$vals[c]\ :=\ 0; \\
\mbox{}\ \ \ \ Var$\_$def[c]\ :=\ \textbf{false}; \\
\mbox{}\ \ \textbf{end}; \\
\mbox{}\ \ Calculate\ :=\ \textbf{DoCalculation}(Root); \\
\mbox{}\textbf{end}; \\
\mbox{} \\
\mbox{}\textit{\{Обработка\ одной\ строки\}} \\
\mbox{}\textbf{procedure}\ \textbf{Evaluate}(S\ :\ string); \\
\mbox{}\textbf{var} \\
\mbox{}\ \ R\ :\ PNode; \\
\mbox{}\ \ Res\ :\ real; \\
\mbox{}\textbf{begin} \\
\mbox{}\ \ R\ :=\ nil; \\
\mbox{}\ \ IsError\ :=\ \textbf{false}; \\
\mbox{}\ \ \textbf{CreateInfixTree}(R,\ S); \\
\mbox{}\ \ \textbf{if}\ (\textbf{not}\ IsError)\ \textbf{then}\ \textbf{begin} \\
\mbox{}\ \ \ \ \textbf{Simplify}(R); \\
\mbox{}\ \ \ \ \textbf{DumpTree}(R,\ 0); \\
\mbox{}\ \ \ \ Res\ :=\ \textbf{Calculate}(R); \\
\mbox{}\ \ \ \ \textbf{if}\ (\textbf{not}\ IsError)\ \textbf{then} \\
\mbox{}\ \ \ \ \ \ \textbf{writeln}(\texttt{'Значение\ выражения\ '},\ Res:8:4) \\
\mbox{}\ \ \textbf{end}; \\
\mbox{}\ \ \textbf{FreeTree}(R); \\
\mbox{}\textbf{end}; \\
\mbox{} \\
\mbox{}\textit{\{Обработка\ файла\}} \\
\mbox{}\textbf{procedure}\ \textbf{ReadFile}(FName\ :\ string); \\
\mbox{}\textbf{var} \\
\mbox{}\ \ F\ :\ \textbf{text}; \\
\mbox{}\ \ S\ :\ string; \\
\mbox{}\textbf{begin} \\
\mbox{}\ \ \textbf{assign}(F,\ FName); \\
\mbox{}\ \ \textbf{reset}(F); \\
\mbox{}\ \ \textbf{while}\ (\textbf{not}\ \textbf{SeekEOF}(F))\ \textbf{do}\ \textbf{begin} \\
\mbox{}\ \ \ \ \textbf{readln}(F,\ S); \\
\mbox{}\ \ \ \ \textbf{Evaluate}(S); \\
\mbox{}\ \ \textbf{end}; \\
\mbox{}\textbf{end}; \\
\mbox{} \\
\mbox{}\textit{\{Отображение\ справки\}} \\
\mbox{}\textbf{procedure}\ ShowHelp; \\
\mbox{}\textbf{begin} \\
\mbox{}\ \ \textbf{writeln}(\texttt{'\ '}:4,\ \texttt{'Помощь:'}); \\
\mbox{}\ \ \textbf{writeln}(\texttt{'\ '}:8,\ \texttt{'F'},\ \texttt{'\ \ \ \ Указать\ файл\ с\ входными\ данными'}); \\
\mbox{}\ \ \textbf{writeln}(\texttt{'\ '}:8,\ \texttt{'S'},\ \texttt{'\ \ \ \ Ввести\ вычисляемую\ строку\ вручную'}); \\
\mbox{}\ \ \textbf{writeln}(\texttt{'\ '}:8,\ \texttt{'H'},\ \texttt{'\ \ \ \ Вывод\ данного\ сообщения'}); \\
\mbox{}\ \ \textbf{writeln}(\texttt{'\ '}:8,\ \texttt{'Q'},\ \texttt{'\ \ \ \ Возврат\ на\ предыдущий\ шаг,\ в\ главном\ меню\ -\ выход'}); \\
\mbox{}\textbf{end}; \\
\mbox{} \\
\mbox{}\textit{\{Запрос\ имени\ файла\}} \\
\mbox{}\textbf{procedure}\ RunFile; \\
\mbox{}\textbf{var} \\
\mbox{}\ \ S\ :\ string; \\
\mbox{}\textbf{begin} \\
\mbox{}\ \ S\ :=\ \texttt{''}; \\
\mbox{}\  \\
\mbox{}\ \ \textbf{repeat} \\
\mbox{}\ \ \ \ \textbf{if}\ (S\ \textless{}\textgreater{}\ \texttt{''})\ \textbf{then} \\
\mbox{}\ \ \ \ \ \ \textbf{writeln}(\texttt{'Файл\ с\ именем\ '},\ S,\ \texttt{'\ не\ существует!'}); \\
\mbox{}\ \ \ \ \textbf{write}(\texttt{'Введите\ имя\ файла\ (пустое\ =\ имя\ по\ умолчанию):\ '}); \\
\mbox{}\ \ \ \ \textbf{readln}(S); \\
\mbox{}\ \ \ \ \textbf{if}\ (S\ =\ \texttt{''})\ \textbf{then} \\
\mbox{}\ \ \ \ \ \ S\ :=\ \texttt{'in.txt'}; \\
\mbox{}\ \ \textbf{until}\ \textbf{FileExists}(S)\ \textbf{OR}\ (\textbf{upcase}(S[1])\ =\ \texttt{'Q'}); \\
\mbox{} \\
\mbox{}\ \ \ \ \textbf{if}\ (\textbf{upcase}(S[1])\ \textless{}\textgreater{}\ \texttt{'Q'})\ \textbf{then} \\
\mbox{}\ \ \ \ \ \ \textbf{ReadFile}(S); \\
\mbox{}\textbf{end}; \\
\mbox{} \\
\mbox{}\textit{\{Обработка\ одной,\ вручную\ введенной\ строки\}} \\
\mbox{}\textbf{procedure}\ RunString; \\
\mbox{}\textbf{var} \\
\mbox{}\ \ S\ :\ string; \\
\mbox{}\textbf{begin} \\
\mbox{}\ \ \textbf{write}(\texttt{'Введите\ выражение:\ '}); \\
\mbox{}\ \ \textbf{readln}(S); \\
\mbox{}\ \ \textbf{if}\ (S\ =\ \texttt{''})\ \textbf{then} \\
\mbox{}\ \ \ \ \textbf{writeln}(\texttt{'Выражение\ не\ должно\ быть\ пустым!'}) \\
\mbox{}\ \ \textbf{else}\ \textbf{if}\ (\textbf{upcase}(S[1])\ \textless{}\textgreater{}\ \texttt{'Q'})\ \textbf{then} \\
\mbox{}\ \ \ \ \textbf{Evaluate}(S); \\
\mbox{}\textbf{end}; \\
\mbox{} \\
\mbox{}\textit{\{Диалог\ с\ пользователем\}} \\
\mbox{}\textbf{procedure}\ MainLoop; \\
\mbox{}\textbf{var} \\
\mbox{}\ \ C\ :\ char; \\
\mbox{}\textbf{begin} \\
\mbox{}\ \ \textbf{writeln}(\texttt{'Программа\ для\ упрощения\ алгебраических\ операций'}); \\
\mbox{}\ \ \textbf{repeat} \\
\mbox{}\ \ \ \ \textbf{write}(\texttt{'Введите\ команду\ (h\ для\ справки):\ '}); \\
\mbox{}\ \ \ \ \textbf{readln}(C); \\
\mbox{}\ \ \ \ \textbf{case}\ \textbf{upcase}(C)\ \textbf{of} \\
\mbox{}\ \ \ \ \ \ \texttt{'H'}:\ ShowHelp; \\
\mbox{}\ \ \ \ \ \ \texttt{'F'}:\ RunFile; \\
\mbox{}\ \ \ \ \ \ \texttt{'S'}:\ RunString; \\
\mbox{}\ \ \ \ \textbf{end}; \\
\mbox{}\ \ \textbf{until}\ \textbf{upcase}(C)\ =\ \texttt{'Q'}; \\
\mbox{}\textbf{end}; \\
\mbox{} \\
\mbox{}\textit{\{Основная\ программа\}} \\
\mbox{}\textbf{begin} \\
\mbox{}\ \ MainLoop; \\
\mbox{}\textbf{end}. \\
\mbox{} \\
\mbox{}
\\
\fbox{Модуль StackUtil:}\\
% Generator: GNU source-highlight, by Lorenzo Bettini, http://www.gnu.org/software/src-highlite
\noindent
\mbox{}unit\ StackUtil; \\
\mbox{} \\
\mbox{}interface \\
\mbox{} \\
\mbox{}\textbf{type} \\
\mbox{}\ \ RType\ =\ (Variable,\ Constant,\ Operation,\ Bracket); \\
\mbox{}\ \ PNode\ =\ \textasciicircum{}TNode; \\
\mbox{}\ \ TNode\ =\ \textbf{record} \\
\mbox{}\ \ \ \ Data\ :\ integer; \\
\mbox{}\ \ \ \ T\ :\ RType; \\
\mbox{}\ \ \ \ Left,\ Right,\ Next\ :\ PNode; \\
\mbox{}\ \ \textbf{end}; \\
\mbox{} \\
\mbox{}\textbf{procedure}\ \textbf{Push}(X\ :\ PNode); \\
\mbox{}\textbf{function}\ Pop\ :\ PNode; \\
\mbox{}\textbf{function}\ ReadTop\ :\ PNode; \\
\mbox{} \\
\mbox{}\textbf{procedure}\ ResetStack; \\
\mbox{}\textbf{function}\ IsClear\ :\ boolean; \\
\mbox{}\textbf{procedure}\ \textbf{Error}(S1,\ S2\ :\ string;\ \textbf{var}\ N\ :\ integer); \\
\mbox{} \\
\mbox{}\textbf{var} \\
\mbox{}\ \ IsError\ :\ boolean; \\
\mbox{} \\
\mbox{}implementation \\
\mbox{} \\
\mbox{}\textbf{const}\ NMax\ =\ 100; \\
\mbox{} \\
\mbox{}\textbf{var} \\
\mbox{}\ \ Stack\ :\ \textbf{array}\ [0..NMax]\ \textbf{of}\ int64; \\
\mbox{}\ \ SP\ :\ integer; \\
\mbox{} \\
\mbox{}\textit{//\ Процедура\ обработки\ ошибок} \\
\mbox{}\textbf{procedure}\ \textbf{Error}(S1,\ S2\ :\ string;\ \textbf{var}\ N\ :\ integer); \\
\mbox{}\textbf{begin} \\
\mbox{}\ \ \textbf{writeln}(\texttt{'Ошибка:\ '},\ S1); \\
\mbox{}\ \ \textbf{writeln}(S2); \\
\mbox{} \\
\mbox{}\ \ \textbf{if}\ (N\ \textless{}\textgreater{}\ 0)\ \textbf{then} \\
\mbox{}\ \ \ \ \textbf{writeln}(\texttt{'\textasciicircum{}'}:N); \\
\mbox{} \\
\mbox{}\ \ \textit{//\ Остановим\ парсер} \\
\mbox{}\ \ N\ :=\ \textbf{length}(S2)\ +\ 1; \\
\mbox{}\ \ IsError\ :=\ \textbf{true}; \\
\mbox{}\textbf{end}; \\
\mbox{} \\
\mbox{}\textit{//\ Пуст\ ли\ стек?} \\
\mbox{}\textbf{function}\ IsClear\ :\ boolean; \\
\mbox{}\textbf{begin} \\
\mbox{}\ \ IsClear\ :=\ SP\ =\ 1; \\
\mbox{}\textbf{end}; \\
\mbox{} \\
\mbox{}\textit{//\ Сбрасывает\ стек} \\
\mbox{}\textbf{procedure}\ ResetStack; \\
\mbox{}\textbf{begin} \\
\mbox{}\ \ \textbf{for}\ SP\ :=\ 0\ \textbf{to}\ NMax\ \textbf{do} \\
\mbox{}\ \ \ \ Stack[SP]\ :=\ 0; \\
\mbox{}\ \ SP\ :=\ 1; \\
\mbox{}\textbf{end}; \\
\mbox{} \\
\mbox{}\textit{//\ Положить\ в\ стек\ элемент} \\
\mbox{}\textbf{procedure}\ \textbf{$\_$Push}(X\ :\ int64); \\
\mbox{}\textbf{begin} \\
\mbox{}\ \ \textbf{if}\ (SP\ \textless{}=\ NMax)\ \textbf{then}\ \textbf{begin} \\
\mbox{}\ \ \ \ Stack[SP]\ :=\ X; \\
\mbox{}\ \ \ \ \textbf{inc}(SP); \\
\mbox{}\ \ \textbf{end}\ \textbf{else}\ \textbf{begin} \\
\mbox{}\ \ \ \ SP\ :=\ 0; \\
\mbox{}\ \ \ \ \textbf{Error}(\texttt{'Переполнение\ стека'},\ \texttt{''},\ SP); \\
\mbox{}\ \ \textbf{end}; \\
\mbox{}\textbf{end}; \\
\mbox{} \\
\mbox{}\textit{//\ Достать\ из\ стека\ элемент} \\
\mbox{}\textbf{function}\ $\_$Pop\ :\ int64; \\
\mbox{}\textbf{begin} \\
\mbox{}\ \ \textbf{if}\ (SP\ \textgreater{}\ 1)\ \textbf{then}\ \textbf{begin} \\
\mbox{}\ \ \ \ \textbf{dec}(SP); \\
\mbox{}\ \ \ \ $\_$Pop\ :=\ Stack[SP]; \\
\mbox{}\ \ \textbf{end}\ \textbf{else}\ \textbf{begin} \\
\mbox{}\ \ \ \ SP\ :=\ 0; \\
\mbox{}\ \ \ \ \textbf{Error}(\texttt{'Попытка\ чтения\ пустого\ стека,\ проверьте\ корректность\ выражения'}, \\
\mbox{}\ \ \ \ \texttt{''},\ SP); \\
\mbox{}\ \ \ \ $\_$Pop\ :=\ 0; \\
\mbox{}\ \ \textbf{end}; \\
\mbox{}\textbf{end}; \\
\mbox{} \\
\mbox{}\textit{//\ Возвращает\ значение\ верхушки\ стека,\ не\ извлекая\ его} \\
\mbox{}\textbf{function}\ $\_$ReadTop\ :\ int64; \\
\mbox{}\textbf{begin} \\
\mbox{}\ \ \textbf{if}\ (SP\ \textgreater{}\ 1)\ \textbf{then}\ \textbf{begin} \\
\mbox{}\ \ \ \ $\_$ReadTop\ :=\ Stack[SP-1]; \\
\mbox{}\ \ \textbf{end}\ \textbf{else}\ \textbf{begin} \\
\mbox{}\ \ \ \ \textbf{writeln}(\texttt{'Stack\ read\ underflow!'}); \\
\mbox{}\ \ \ \ $\_$ReadTop\ :=\ 0; \\
\mbox{}\ \ \textbf{end}; \\
\mbox{}\textbf{end}; \\
\mbox{} \\
\mbox{}\textit{//\ Обертки} \\
\mbox{}\textbf{procedure}\ \textbf{Push}(X\ :\ PNode); \\
\mbox{}\textbf{begin} \\
\mbox{}\ \ \textbf{$\_$Push}(\textbf{Int64}(X)); \\
\mbox{}\textbf{end}; \\
\mbox{} \\
\mbox{}\textbf{function}\ Pop\ :\ PNode; \\
\mbox{}\textbf{begin} \\
\mbox{}\ \ Pop\ :=\ \textbf{Pointer}($\_$Pop); \\
\mbox{}\textbf{end}; \\
\mbox{} \\
\mbox{}\textbf{function}\ ReadTop\ :\ PNode; \\
\mbox{}\textbf{begin} \\
\mbox{}\ \ ReadTop\ :=\ \textbf{Pointer}($\_$ReadTop); \\
\mbox{}\textbf{end}; \\
\mbox{} \\
\mbox{}\textit{//\ Инициализация\ стека} \\
\mbox{}\textbf{begin} \\
\mbox{}\ \ ResetStack; \\
\mbox{}\textbf{end}. \\
\mbox{} \\
\mbox{}

\pagebreak

\begin{thebibliography}{99}

\bibitem{plaksin}
  Плаксин М.А.,
  \emph{Тестирование и отладка},
  Москва, Бином,
  2009
\bibitem{korolev}
  Королев Л.Н, Миков А.И.,
  \emph{Информатика. Введение в компьютерные науки},
  Москва, Арбис,
  2014
\bibitem{wiki}
  Wikipedia,
  \emph{http://ru.wikipedia.org/wiki/Связные\_списки}
\end{thebibliography}
\pagebreak

\end{document}
