\section{Детальное описание алгоритма}
\DeclareRobustCommand{\l}[1]{
\item {[}{\it #1}{]} 
}
Алгоритм состоит из следующих основных процедур:
\begin{itemize}
\item TMainForm.FormCreate - задает начальные параметры программы, считывая
их, если это возможно, из входного файла
\item TMainForm.FormPaint - прорисовывает в окне программы все основные
элементы, как-то поле, коня и траву
\item TMainForm.FieldPaintBoxMouseDown - обработка события нажатия кнопки мыши
на поле
\item BuildHorseMatrix2 - построение матрицы графа, инициализация + прямой ход
алгоритма
\item ReverseWave - обратный ход алгоритма 
\item TMainForm.FormClose - освобождение ресурсов и запись результата в
выходной файл
\end{itemize}

Детальное описание:
\begin{enumerate}
\item При запуске программы вызывается TMainForm.FormCreate. Она подгружает
графические ресурсы с диска и пытается прочитать начальные значения параметров
из входного файла. Если считывание не удалось, или во время него произошла
ошибка, параметрам присваиваются значения по умолчанию.
\item Основной цикл работы программы включает в себя два главных компонента:
отображение результата работы алгоритма и обработку действий пользователя
\begin{enumerate}
\item За отображение результата отвечает процедура TMainForm.FormPaint. Она
прорисовывает в окне прогаммы необходимые компоненты:
\begin{enumerate}
\item Сначала рисуется поле $NxN$ клеток-квадратов, клетки раскрашены в цвет
шахматной доски
\item Затем поверх поля отображается текущая позиция коня и травы, конь и
трава - спрайты, подгружаемые процедурой FormCreate
\item После этого рисуется текущий путь коня, начиная с его позиции и
заканчивая позицией, на которой размещена трава. Путь отображается
последовательностью символов ``галочка'', этот символ также является спрайтом
\item В самом конце производится проверка, нужно ли отображать матрицу
промежуточных результатов; если да - поверх полученного ранее изображения
рисуется матрица (для каждой клетки - расстояние от нее до клетки с конем)
\end{enumerate}
\item За взаимодействие с пользователем отвечает процедура
TMainForm.FieldPaintBoxMouseDown:
\begin{enumerate}
\item Если произошел щелчок левой кнопкой мыши, вычисляем новые координаты
коня, и, если они допустимы (не выходят за пределы поля), переносим его на
новую позицию
\item Если произошел щелчок правой кнопкой, то вычисляем новые координаты
травы и, соответственно, при их корректности переносим траву
\item По окончании переноса коня и травы заново запускаем алгоритм и форсируем
перерисовку
\end{enumerate}
\item Также на форме имеется элемент для ввода размерности поля. При его
изменении меняется размерность поля, а также корректируются координаты коня и
травы, если они выходят за границы поля новой размерности. По окончании
обработки события форсируется обновление данных и перерисовка поля.
\item По завершении работы программы результирующие данные (количество ходов
коня и координаты клеток, по которым пройдет его маршрут) записываются в
результирующий файл.
\end{enumerate}
\end{enumerate}

\pagebreak

