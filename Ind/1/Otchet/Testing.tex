\section{Тестирование программы}
\newcounter{testcnt}
\DeclareRobustCommand{\test}[4]{
  \addtocounter{testcnt}{1}
  \par
  Тест \arabic{testcnt}: {#1}\\
  {\it Входные данные:} {#2}\\
  {\it Выходные данные:} {#3}\\
  {\it Ожидаемый результат:} {#4}\\
  \begin {figure}[H] 
  \centering 
  \includegraphics[scale=0.5]{test\arabic{testcnt}.png} 
  \caption{Скриншот} 
  \label{fig:scr\arabic{testcnt}} 
  \end {figure}
}
\par
Тесты для программы были разбиты на несколько групп таким образом, чтобы
покрыть одновременно максимальное количество типов входных данных и вариантов
выполнения тела программы.\\
%\test{}{}{}{}
\test{Проверка меню}{Клавиша помощи ``H''}{Экран помощи}{Экран помощи}
\test{Некорректный ввод}{Несуществующее имя файла}{Сообщение об отсутствии
файла}{Сообщение об отсутствии файла}
\test{Корректный ввод}{Имя файла с корректными исходными данными}{Успешное
завершение}{Успешное завершение}
\test{Некорректный ввод}{Ошибка в выражении - многобуквенная
переменная}{Сообщение об ошибке}{Сообщение об ошибке}
\test{Некорректный ввод}{Ошибка в выражении - нечисловой показатель}{Сообщение
об ошибке}{Сообщение об ошибке}
\test{Некорректный ввод}{Ошибка в выражении - пропущен показатель}{Сообщение
об ошибке}{Сообщение об ошибке}
\test{Некорректный ввод}{Ошибка в выражении - свободный член оказался не в
конце выражения}{Сообщение об ошибке}{Сообщение об ошибке}
\pagebreak
