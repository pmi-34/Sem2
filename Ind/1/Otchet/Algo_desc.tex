\section{Детальное описание алгоритма}
\DeclareRobustCommand{\l}[1]{
\item {[}{\it #1}{]} 
}
Алгоритм состоит из следующих основных процедур:
\begin{itemize}
\item main - организует приглашение командной строки, по запросу пользователя
запускает задачу, показывает справку или завершает работу программы
\item Help - демонстрирует экран помощи
\item RunDialog - запрашивает у пользователя имя файла и в случае его
корректности запускает задачу - Run
\item Run - основная задача, считывает из файла два многочлена и производит
над ними требуемые действия
\item ParseP - производит разбор записанного многочлена во внутренний формат
программы
\item FormP - формирует строковое представление одного члена многочлена
\item MergeP - сливает два списка в третий, учитывая производимое действие
(сложение/вычитание)
\item Error - отображает сообщение об ошибке
\end{itemize}

Детальное описание:
\begin{enumerate}
  \l{main} Исполнение программы начинается с функции main
  \l{Greet} Выводим на экран приветствие
  \l{RunDialog} Запускаем главный цикл:
  \begin{enumerate}
    \l{RunDialog} Считываем строку, проверяем файл на существование
    \l{Run} Открываем файл, считываем и разбираем два многочлена. Разбор
    происходит рекурсивным способом.
    \begin{enumerate}
      \l{ParseP} Для каждого члена процедура выполняется рекурсивно
      \l{ParseP} Если при разборе произошла ошибка, то разбор останавливается
      и выдается соответствующее сообщение
    \end{enumerate}
    \l{Run} Если преобразование прошло успешно:
    \begin{enumerate}
      \l{AddP, SubP} Вычисляем сумму и разность двух многочленов:
      \begin{enumerate}
        \l{MergeP} В случае суммы члены обоих многочленов имеют коэффициент
        при слиянии, равный единице 
        \l{MergeP} В случае разности первый имеет коэффициент единицу, а
        второй - минус единицу
      \end{enumerate}
      \l{Run} Записываем полученные значения в файл, рапортуем пользователю об
      успехе
    \end{enumerate}
    \l{main} Запрашиваем у пользователя подтверждение на продолжение
    сеанса взаимодействия с программой
    \l{main} Если пользователь ответил отрицательно, то выходим из
    цикла
  \end{enumerate}
  \l{main} Завершение работы программы
\end{enumerate}

\pagebreak

