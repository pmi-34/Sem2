\section{Теория}
\DeclareRobustCommand{\c}[1]{
\begin{center}
{#1}
\end{center}
}
\par
{\bf Связный список} — базовая динамическая структура данных, состоящая из 
узлов, каждый из которых содержит как собственно данные, так и одну или две
 ссылки (``связки'') на следующий и/или предыдущий узел списка. 
Принципиальным преимуществом перед массивом является структурная гибкость: 
порядок элементов связного списка может не совпадать с порядком расположения 
элементов данных в памяти компьютера, а порядок обхода списка всегда явно 
задаётся его внутренними связями.

{\bf Линейный однонаправленный список} - это структура данных, состоящая 
из элементов одного типа, связанных между собой единственной связью -
от предыдущего к следующему.
На практике линейные списки обычно реализуются при помощи массивов и 
связных списков. Иногда термин ``список'' неформально используется 
также как синоним понятия ``связный список''.

Характеристики:

\begin{enumerate}
  \item {\bf Длина списка}. Количество элементов в списке;
  \item Списки могут быть {\bf типизированными} или {\bf нетипизированными}. 
        Если 
        список типизирован, то тип его элементов задан, и все его элементы 
        должны иметь типы, совместимые с заданным типом элементов списка;
        Обычно списки являются типизированными;
  \item Список может быть {\bf сортированным} или {\bf несортированным};
  \item В зависимости от реализации может быть возможен {\bf произвольный доступ} к 
        элементам списка.
\end{enumerate}
\pagebreak
