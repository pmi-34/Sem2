\section{Определение идеи алгоритма, выбор методов решения и структур данных}
\par
Перед программой стоит задача - считать из файла два многочлена, найти их
сумму и разность, после чего дописать результаты вычисления в тот же самый
файл.
\par
Для начала определимся с типами данных. По условию задачи необходимо
использовать односвязный линейный список, каждый элемент которого, кроме ссылки
на следующий содержит в себе еще и два информационных поля - степень данного
члена в многочлене и коэффициент при нем. Также я счел целесообразным добавить
еще одно поле символьного типа - собственно обозначение переменной (x, y, z, t 
и тп). Данное поле корректно обрабатывается процедурами ввода/вывода, но при
арифметических действиях содержимое его не учитывается; оно было введено с
учетом возможного расширения
функциональности в будущем. Для хранения значений коэффициентов и показателей
степени был выбран тип {\bf integer}. 
\par
Ввод многочленов по условию организован через файл; примем во внимание, что
человеку свойственно ошибаться, и предусмотрим при разборе многочлена из
считанной строки проверку
на корректность выражения.
\par
Программа будет состоять из трех модулей - модуль ввода данных, модуль их
преобразования и модуль вывода. В модуле ввода центральное место займет
небольшой анализатор выражений; в модуле вывода - симметричный ему алгоритм
конвертации внутреннего представления в строку. В модуле преобразования
ключевую роль сыграет алгоритм слияния двух списков с учетом значений их
информационных полей и текущей выполняемой операции (сложения или вычитания).
Собственно процедуры сложения/вычитания будут уже опираться на данный
алгоритм.
\par
Пару слов насчет интерфейса. Поскольку основной ввод/вывод организуется через
файл, у пользователя необходимо запросить только лишь его имя (и проверить на
существование). Но в случае ошибки было бы неплохо указать, где именно
споткнулась программа (а такое место может быть только одно - анализатор
выражений), и в любом случае - вывести какое-нибудь сообщение по окончании
работы программы (``Все хорошо, хозяин'' или ``Явка провалена из-за ошибки
резидента'').
\pagebreak

